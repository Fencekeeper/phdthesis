\abstractintoc % Add abstract to Table of Contents  
%\abstractnum  % Format abstract like a chapter

\begin{abstract}

\noindent A simplicial set is said to be \textbf{non-singular} if its non-degenerate simplices are embedded. Let $sSet$ denote the category of simplicial sets. We prove that the full subcategory $nsSet$ whose objects are the non-singular simplicial sets admits a model structure such that $nsSet$ becomes is Quillen equivalent to $sSet$ equipped with the standard model structure due to Quillen \cite{Qu67}. The model structure on $nsSet$ is right-induced from $sSet$ and it makes $nsSet$ a proper cofibrantly generated model category. Together with Thomason's model structure on small categories \cite{Th80} and Raptis' model structure on posets \cite{Ra10} these form a square-shaped diagram of Quillen equivalent model categories in which the subsquare of right adjoints commutes.










\noindent Simplicial sets has a standard model structure due to Quillen in which the weak equivalences are the morphisms whose geometric realizations are weak homotopy equivalences and in which the fibrations are the Kan fibrations. A simplicial set is non-singular if its non-degenerate simplices are embedded. The inclusion of the full subcategory of non-singular simplicial sets into the category of simplicial sets has a left adjoint, referred to under the name desingularization.

The Kan subdivision has a right adjoint, called extension. The double subdivision followed by desingularization is then left adjoint to the inclusion followed by the double extension. Inspired by Thomason \cite{Th80}, we lift the standard model structure on simplicial sets to non-singular simplicial sets by letting a morphism be a weak equivalence (fibration) whenever it becomes a weak equivalence (fibration) after applying the inclusion followed by the double extension. Thus non-singular simplicial sets is a proper cofibrantly generated (closed) model category and the double (Kan) subdivision followed by desingularization is the left Quillen functor of a Quillen equivalence.

The category of non-singular simplicial sets is of interest in piecewise linear topology, and sits strictly between ordered simplicial complexes and simplicial sets. As a model for spaces, the older (ordered) simplicial complexes are more directly put into the modern context of model categories by our model structure on non-singular simplicial sets.

A difficulty when attempting to lift the model structure on simplicial sets to the non-singular ones is that the only known description is non-concrete. Part of this thesis is to work around that. A noteworthy result in that regard is that the operation of taking a product with a non-singular simplicial set is an endofunctor of non-singular simplicial sets that preserves colimits.
    
Asking for a characterization of the cofibrant objects in the obtained model structure on non-singular simplicial sets leads to whether the Barratt nerve is the optimal triangulation of regular simplicial sets. Put precisely, one is lead to ask whether the natural map from the desingularized subdivision to the Barratt nerve is an isomorphism in the case when the simplicial set is regular. Waldhausen expects an affirmative answer to this question, which is put forward by Waldhausen, Jahren and Rognes \cite{WJR13}. Indeed, our third major result is a confirmation of Waldhausen's suspicion. As a consequence, the improvement functor used in \cite{WJR13} to make simplicial sets non-singular is not at all ad hoc and has even better properties than previously known. The improvement functor is by definition Kan subdivision followed by the Barratt nerve.
\end{abstract}



