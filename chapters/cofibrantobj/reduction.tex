\section{Reduction of the problem}
\label{sec:red}

We aim to prove that the natural map $t_X:DSd\, X\to BX$ is an isomorphism for regular $X$. We will begin by reducing the problem in two steps.

The skeleton filtration of an arbitrary simplicial set $X$ gives rise to the diagram
\begin{displaymath}
\xymatrix{
DSd\, X^0 \ar[d]^{t_{X^0}} \ar[r] & DSd\, X^1 \ar[d]^{t_{X^1}} \ar[r] & \dots \ar[r] & DSd\, X^n \ar[d]^{t_{X^n}} \ar[r] & \dots \\
BX^0 \ar[r] & BX^1 \ar[r] & \dots \ar[r] & BX^n \ar[r] & \dots
}
\end{displaymath}
and if the vertical maps are all isomorphisms, then $t_X$ is. This is due to the following three facts. First, the functor $DSd$ is a left adjoint. Second, the functor $(-)^\sharp :sSet\to PoSet$ is cocontinous as the natural map $pcSd\, X\to X^\sharp$ is an isomorphism. Third, the functors $U:PoSet\to Cat$ and $N:Cat\to sSet$ preserve sequential colimits. \todo{Add reference and check what generality this result allows.}

The map $t_{X^0}$ is an isomorphism for any $X$ as $b_{X^0}$ is. Note that the $n$-skeleton $X^n$ can be built (transfinitely) from $X^{n-1}$ by successively attaching the non-degenerate $n$-simplices along their boundaries. We suggest the following problem reduction to begin with.
\begin{lemma}\label{lem:first_reduction}
The natural map $t_X:DSd\, X\to BX$ is an isomorphism for regular simplicial sets $X$ if it is an isomorphism for any regular $X$ that is generated by a single simplex.
\end{lemma}
\begin{proof}
The map $t_X$ is an isomorphism for any $0$-dimensional simplicial set $X$ as $b_X$ is an isomorphism in that case. Now, consider regular simplicial sets of dimension at most $n>0$. We will prove the statement of \cref{lem:first_reduction} by an induction whose hypothesis will be the following.

Assume that $\lambda$ is an ordinal such that $t_X$ is an isomorphism whenever $X$ is regular and the colimit of some $\gamma$-sequence
\begin{displaymath}
\xymatrix{
X^{n-1}=Y^0 \ar[r] & Y^1 \ar[r] & \dots \ar[r] & Y^\beta \ar[r] & \dots \; ,
}
\end{displaymath}
$\gamma <\lambda$, that begins at the $(n-1)$-skeleton $X^{n-1}$ of $X$ and whose maps $Y^\beta \to Y^{\beta +1}$, $\beta +1<\gamma$, fit into squares
\begin{displaymath}
\xymatrix{
\partial \Delta [n] \ar[d] \ar[r] & Y^\beta \ar[d] \\
\Delta [n] \ar[r] & Y^{\beta +1}
}
\end{displaymath}
that are cocartesian (in $sSet$), or in other words are cobase changes of attachings of $n$-simplices.

Under the induction hypothesis, the map $t_X$ is an isomorphism when $X$ is regular and of dimension $n-1$ or lower, for $X$ is in that case the colimit of the $1$-sequence
\begin{displaymath}
\xymatrix{
X=X^{n-1}=Y^0\; .
}
\end{displaymath}

For the base step of the induction, we note that the hypothesis is true for $\lambda =0$ as a $0$-sequence is simply the empty diagram, which has the empty simplicial set as its colimit.

For the inductive step, suppose $X$ is regular and the colimit of a $\lambda$-sequence that fits the description given above. The ordinal $\lambda$ is either a limit ordinal or a successor ordinal.

If $\lambda$ is a limit ordinal, then $t_X$ arises from the diagram
\begin{displaymath}
\xymatrix{
DSd\, X^{n-1}=DSd\, Y^0 \ar[d]^{t_{Y^0}} \ar[r] & DSd\, Y^1 \ar[d]^{t_{Y^1}} \ar[r] & \dots \ar[r] & DSd\, Y^n \ar[d]^{t_{Y^n}} \ar[r] & \dots \\
BX^{n-1}=BY^0 \ar[r] & BY^1 \ar[r] & \dots \ar[r] & BY^n \ar[r] & \dots \; .
}
\end{displaymath}
It follows that $t_X$ is an isomorphism by the same argument as we used when arguing that $t_X$ is an isomorphism whenever $t_{X^n}$ is an isomorphism for any skeleton $X^n$.

If $\lambda =\beta +1$ is a successor ordinal it is far from obvious that $t_X$ is an isomorphism. Denote $X'=Y^\beta$. Then $X$ is a pushout
\begin{displaymath}
\xymatrix{
\partial \Delta [n] \ar[d] \ar[r] & X' \ar[d] \\
\Delta [n] \ar[r]_{\bar{x} } & X
}
\end{displaymath}
meaning $X'$ is the final building stage of $X$ before $x$ is attached. By definition, the regular simplicial set $X'$ is the colimit of a $\beta$-sequence, so $t_{X'}$ is an isomorphism by the induction hypothesis.

In general, the Barratt nerve behaves badly when applied to pushouts, so we now decompose $X$ in a different way. The following decomposition does not depend on regularity, though $X$ is regular. Let $Y$ denote the simplicial subset of $X$ that is generated by $x$, or in other words, the image of $\bar{x}$. If we take the pullback along the inclusion $X'\to X$ we get a diagram
\begin{displaymath}
\xymatrix{
\partial \Delta [n] \ar[dd] \ar@{-->}[dr] \ar[rr] && X' \ar[dd] \\
& Y' \ar[dd] \ar[ur] \\
\Delta [n] \ar[dr] \ar@{-}[r]^(.65){\bar{x} } & \ar[r] & X \\
& Y \ar[ur]
}
\end{displaymath}
that gives rise to a factorization
\[X\to Y\sqcup _{Y'}X'\to X\]
of the identity. It is not hard to realize that $Y\sqcup _{Y'}X'\to X$ is degreewise injective. In other words, the simplicial set $X$ can be viewed as the pushout $Y\sqcup _{Y'}X'$.

As the dimension of $Y'$ is at most $n-1$ we trivially get that $t_{Y'}$ is an isomorphism, so we have the diagram
\begin{displaymath}
\xymatrix{
DSd\, Y \ar[d]_{t_Y} & DSd\, Y' \ar[l] \ar[d]_{t_{Y'}}^\cong \ar[r] & DSd\, X' \ar[d]_{t_{X'}}^\cong \\
B\, Y & B\, Y' \ar[l] \ar[r] & B\, X'
}
\end{displaymath}
giving rise to a map between pushouts in $nsSet$ that $t_X$ factors through. In fact, $t_X$ \emph{is} (identified with) the arising map between pushouts as we explain in the next paragraph. This means that $t_X$ is an isomorphism if $t_Y$ is. A non-degenerate $n$-simplex generates $Y$, thus the next paragraph finishes the proof.

The sharp functor $(-)^\sharp :SSet\to PoSet$ preserves colimits, so the diagram
\begin{displaymath}
\xymatrix{
(Y')^\sharp \ar[d] \ar[r] & (X')^\sharp \ar[d] \\
Y^\sharp \ar[r] & X^\sharp
}
\end{displaymath}
is cocartesian in $PoSet$. Moreover, $(-)^\sharp$ turns degreewise injective maps into sieves. In fact, the square is cocartesian when considered as a diagram in $Cat$. The nerve $N:Cat\to sSet$ preserves the pushout $X^\sharp$, which says that the Barratt nerve $B:sSet\to sSet$ preserves the pushout $Y\sqcup _{Y'}X'$. As the pushout
\[BX=BY\sqcup _{BY'}BX'\]
in $sSet$ and the three other objects in that square-shaped diagram are all non-singular implies that the diagram is in fact cocartesian in $nsSet$ also.
\end{proof}
\noindent The purpose of reducing the proof that $t_X$ is an isomorphism for regular $X$ to the case when $X$ is generated by a single simplex is that we can then argue in terms of mapping cylinders. Actually, it is the names and notation of mapping cylinders that we will take advantage of. It is really Dwyer maps that we will use when arguing. Still, we would like to remind the reader of the three mapping cylinders that are relevant in our setting.

Recall that simplicial maps $f$ between arbitrary simplicial sets have a non-singular mapping cylinder $M(f)$, called the reduced mapping cylinder \cite{WJR13}. Like the topological mapping cylinder $T(f)$ it comes with a front inclusion $X\to M(f)$ and a back inclusion $Y\to M(f)$, and these are compatible with a canonically defined natural reduction map $T(f)\to M(f)$. If $f=N\varphi$ is the nerve of an order-preserving function $\varphi :P\to R$, then the reduction map $T(N\varphi )\to M(N\varphi )$ is identified with the canonical map
\[NP\times \Delta [1]\sqcup _{NP}NR\to N(P\times [1]\sqcup _PR).\]
from a pushout in $sSet$ to the nerve of a corresponding pushout in $PoSet$. Finally, there is the desingularized topological mapping cylinder, which is simply the desingularization of $T(f)$.

Suppose $n$ is such that $t_X:DSd\, X\to BX$ is an isomorphism whenever $X$ is regular and generated by a non-degenerate $k$-simplex for some $k<n$. For the base step one can note that the hypothesis is satisfied for $n=1$. The induction step is handled by the following result, which is our second problem reduction.
\begin{lemma}\label{lem:second_reduction}
Suppose $X$ is a regular simplicial set that is generated by a non-degenerate $n$-simplex $x$, which implies that it is the pushout in the diagram
\begin{displaymath}
\xymatrix{
\Delta [n-1] \ar[d]_{N\delta _n} \ar[r]^(.6){\bar{y} } & Y \ar[d] \\
\Delta [n] \ar[r]_{\bar{x} } & X
}
\end{displaymath}
where we have written $y=x\delta _n$. Denote $P=\Delta [n-1]^\sharp$, $Q=\Delta [n]^\sharp$ and $R=Y^\sharp$. The pushout $Q\sqcup _PR$ in $Cat$ is a poset, so the simplicial set $N(Q\sqcup _PR)$ is non-singular. Furthermore, ...
\begin{enumerate}
\item{
...the map $t_X:DSd\, X\to BX$ is an isomorphism if the canonical map
\[D(NQ\sqcup _{NP}NR)\to N(Q\sqcup _PR)\]
is.
}
\item{
...the map $D(NQ\sqcup _{NP}NR)\to N(Q\sqcup _PR)$ is an isomorphism if the map
\[DT(B(\bar{y} ))\xrightarrow{\zeta } M(B(\bar{y} ))\]
from the desingularized topological mapping cylinder of $B(\bar{y} )$ to the reduced mapping cylinder is an isomorphism.
}
\end{enumerate}
\end{lemma}
\noindent Here, the small category $Q\sqcup _PR$ denotes the pushout in $Cat$ of a diagram that consists only of posets. According to Thomason's result Lemma 5.6.4. of \cite{Th80}, the pushout is a poset as $P\to Q$ is Dwyer. We will point out the precise Dwyer structure, shortly. Because $Q\sqcup _PR$ is a poset, a map 
\[D(NQ\sqcup _{NP}NR)\to N(Q\sqcup _PR)\]
arises from
\[NQ\sqcup _{NP}NR\to N(Q\sqcup _PR).\]
The former of these two is the one referred to by \cref{lem:second_reduction}.

We will verify Part $1$ of the lemma immediately as it is straight forward. Part $2$ requires a little more work, of which the key part is a technique barrowed from the proof of Proposition 4.3 in \cite{Th80}. Desingularization and non-singular simplicial sets are uncommon in the literature, but a very limited knowlegde of them is sufficient in the the following proof of \cref{lem:second_reduction}. Our argument involves understanding pushouts of special diagrams in the four categories $sSet$, $nsSet$, $PoSet$ and $Cat$, which are closely related as we have seen in the diagram \cref{fig:Square_adjunctions}..

To factor $t_X$ in a useful way, we begin by factoring $b_X$ by means of the diagram
\begin{displaymath}
\xymatrix{
& Sd(\Delta [n]) \ar@{-}[d] \ar[dr] && Sd(\Delta [n-1]) \ar[ll]_{Sd(N\delta _n)} \ar@{-}[d] \ar[dr] \\
& \ar[d]_(.3)\cong ^(.3)b & Sd\, X \ar@/_6.3pc/[lldddd]_b \ar[dd]_(.65)f & \ar[d]_(.3)\cong ^(.3)b & Sd\, Y \ar[ll] \ar[dd] ^b \\
& B(\Delta [n]) \ar@/_/[lddd] \ar@/_/[dd] \ar[dr] & \ar[l] & B(\Delta [n-1]) \ar@{-}[l]_(.7){B(N\delta _n)} \ar[dr] \\
&& X' \ar[ld] && BY \ar@/^/[llld] \ar[ll] \ar@/^1pc/[lllldd] \\
& N(Q\sqcup _PR) \ar[ld] \\
BX
}
\end{displaymath}
in $sSet$. Here, the lower square of the cube is cocartesian and we have written $X'=NQ\sqcup _{NP}NR$ for brevity. We have abbreviated by simply writing the letter $b$ for the natural map from the normal subdivision to the Barratt nerve.

We begin by pointing out that the canonical map
\[Q\sqcup _PR\xrightarrow{\cong } X^\sharp\]
is an isomorphism. The reason for this is three-fold. First, the functor $(-)^\sharp :sSet\to PoSet$ is cocontinous. Second, the pushout $Q\sqcup _PR$ taken in $Cat$ is a poset. Third, $PoSet$ is a reflective subcategory of $Cat$.

Naturality of $d_{Sd\, X}$ yields the diagram
\begin{displaymath}
\xymatrix{
Sd\, X \ar[d]_f \ar[r]^(.42)d & DSd\, X \ar[d]^{D(f)} \\
X' \ar@/_3pc/[dd]_{k} \ar[d] \ar[r]^(.47)d & DX' \ar@{-->}[ldd]_(.53)l \ar[d] \ar@/^4pc/[dd]^{D(k)} \\
N(Q\sqcup _PR) \ar[d]_\cong & DN(Q\sqcup _PR) \ar[d] \\
BX \ar[r]_(.45)d^(.45)\cong & DB\, X
}
\end{displaymath}
in which the diagonal map $l$ of the lower square arises due to the universal property of desingularization. It makes the upper left triangle of the lower square commute. Then the lower right triangle of the lower square commutes, also. This means we have a factorization of
\[b_X=k\circ f=l\circ d_{X'}\circ f=l\circ D(f)\circ d_{Sd\, X}\]
through $d_X$. The map $t_X$ is unique, so it follows that
\[t_X=l\circ D(f).\]
The map $l$ factors through $DX'\to N(Q\sqcup _PR)$ and the latter is an isomorphism if and only if the former is.

To prove Part $1$ of \cref{lem:second_reduction} it only remains to argue that $D(f)$ is an isomorphism, which we will do now. The argument is strictly formal. The map $D(f)$ is the canonical map between pushouts of $nSSet$ as $f$ is, by the universal property. It can be factored by applying the cocontinous functor $D$ to the diagram
\begin{displaymath}
\xymatrix{
Sd(\Delta [n-1]) \ar@/_5pc/[dddd]^b_\cong \ar[dd]_\cong ^d \ar[dr] \ar[rr] && Sd\, Y \ar@{-}[d] \ar[dr] \\
& Sd(\Delta [n]) \ar[dd]_(.6)\cong ^(.6)d \ar[rr] & \ar[d]^(.3)d & Sd\, X \ar[dd]_g \ar@/^2pc/[dddd]^f \\
DSd(\Delta [n-1]) \ar[dd]_\cong ^t \ar[dr] \ar@{-}[r] & \ar[r] & DSd\, Y \ar@{-}[d] \ar[dr] \\
& DSd(\Delta [n]) \ar[dd]_(.6)\cong ^(.6)t \ar[rr] & \ar[d]^(.3)t _(.3)\cong & X'' \ar[dd]_h \\
B(\Delta [n-1]) \ar[dr] \ar@{-}[r] & \ar[r] & BY \ar[dr] \\
& B(\Delta [n]) \ar[rr] && X'
}
\end{displaymath}
in $sSet$. The map $D(g)$ is an isomorphism because it is the canonical map between pushouts in $nSSet$ and because its source $Sd\, X$ and target $X''$ are the most obvious ways of forming the pushout of the same diagram. The map $t_Y$ is an isomorphism by the induction hypothesis as $Y$ is regular and generated by $y^\sharp$, which is of degree less than or equal to $n-1$. It follows that $D(h)$ is an isomorphism, hence $D(f)$ is an isomorphism. Therefore, the map $t_X$ is an isomorphism if $l$ is. This finishes the proof of Part $1$.

We are about to prove Part 2 of the \cref{lem:second_reduction}. The argument relies upon use of the factorization
\begin{displaymath}
\xymatrix{
P \ar[dr]_{i_0} \ar[rr]^{N(\delta _n)^\sharp } && Q \\
& W=\Delta [n-1]^\sharp \times [1] \ar[ur]_\psi
}
\end{displaymath}
of $(-)^\sharp$ of $N(\delta _n):\Delta [n-1]\to \Delta [n]$. Here, $\psi$ sends the pair $(x:[m]\to [n-1],0)$ to the composite
\[[m]\xrightarrow{x} [n-1]\xrightarrow{\delta _n} [n],\]
whereas the pair $(x:[m]\to [n-1],1)$ is sent to the operator
\[[m+1]\to [n]\]
given by $j\mapsto x(j)$ for $0\leq j\leq m$ and $m+1\mapsto n$. This factorization confirms that $N(\delta _n)^\sharp$ is a Dwyer map, which is what guarantees that the pushout $Q\sqcup P_R$ in $Cat$ is a poset. Notice that there is only one object of $Q$ that is not in the image of $\psi$, namely the $n$-th vertex $\varepsilon _n:[0]\to [n]$.

Now we tie the study of mapping cylinders of $N\varphi$ to the study of $DX'\to N(Q\sqcup _PR)$ by means of the diagram
\begin{displaymath}
\xymatrix{
NP \ar[dd]^{Ni} \ar[dr] \ar[rr]^{N\varphi } && NR \ar@{-}[d] \ar[dr] \\
& NR \ar[dd] & \ar[d] & NR \ar[ll] \ar[dd] \\
NW \ar[dd]^{Nj} \ar[dr] \ar@{-}[r] & \ar[r] & NW\sqcup _{NP}NR \ar@{-}[d] \ar[dr]^\eta \\
& N(W\sqcup _PR) \ar[dd] & \ar[d] & D(NW\sqcup _{NP}NR) \ar[ll]_\zeta \ar[dd] \ar@/^6.5pc/[dddd] \\
NQ \ar[dr] \ar@{-}[r] & \ar[r] & NQ\sqcup _{NP}NR \ar@{-}@/_/[d] \ar[dr]^{\bar{\eta } } \\
& N(Q\sqcup _PR) & \ar@/_1pc/[ddr]^\eta & NQ\sqcup _{NW}D(NW\sqcup _{NP}NR) \ar[ll]_{\bar{\zeta }} \ar@{-->}[dd]_\xi \\
\\
&&& D(NQ\sqcup _{NP}NR) \ar@/^1pc/[lluu]^{\hat{\zeta } }
}
\end{displaymath}
which lives in $sSet$. The pushout $W\sqcup _PR$ is formed in $Cat$, but as $i$ is Dwyer, this category is a poset. Hence $T(N\varphi )=NW\sqcup _{NP}NR$ is the topological mapping cylinder of $N\varphi$, and $M(N\varphi )=W\sqcup _PR$ is the reduced one. Recall that $PoSet$ is a reflective subcategory of $Cat$.

Part 2 of \cref{lem:second_reduction} will follow readily if we can prove the following.
\begin{lemma}\label{lem:proof_of_second_reduction}
The map $\hat{\zeta }$ is cobase change (in $sSet$) of $\zeta$.
\end{lemma}
\noindent If this lemma is correct, then $\hat{\zeta }$ is an isomorphism if $\zeta$ is. The latter lemma is correct if
\begin{displaymath}
\xymatrix{
NW \ar[d]_{Nj} \ar[r] & N(W\sqcup _PR) \ar[d] \\
NQ \ar[r] & N(Q\sqcup _PR)
}
\end{displaymath}
is cocartesian (in $sSet$) and if $\xi$ is an isomorphism. Now we verify that these two conditions hold.

Let $V$ be the full subcategory of $Q$ whose objects are those that are not in $P$. The square above fits into the bigger diagram
\begin{displaymath}
\xymatrix{
& NW \ar@{-}[d] \ar[dr] \\
NV\cap NW=N(V\cap W) \ar[dd] \ar[ur] \ar[rr] & \ar[d]^(.35){Nj} & N(W\sqcup _PR) \ar[dd] \\
& NQ \ar[dr] \\
NV \ar[rr] \ar[ur] && N(Q\sqcup _PR)
}
\end{displaymath}
which verifies that the former square is cocartesian in the following way. By definition, $V$ is a cosieve in $Q$. Furthermore, $V\cap W\to V$ and $V\cap W\to W$ are cosieves, so it follows that we have decomposed $Q$ as a pushout in $Cat$. Observe that $V\cap W\to W\sqcup _PR$ is also a cosieve. It follows that $N:CAT\to SSet$ preserves the pushouts $Q$ and $Q\sqcup _PR$.

The nerve of the cosieve $V\cap W\to W\sqcup _PR$ factors through
\[NV\cap NW\to D(NW\sqcup _{NP}NR),\]
so the latter is degreewise injective. Therefore
\[NQ\sqcup _{NW}D(NW\sqcup _{NP}NR)\cong NV\sqcup _{NV\cap NW}D(NW\sqcup _{NP}NR)\]
is non-singular. This shows that we get a dashed map in the lower left triangle of the diagram
\begin{displaymath}
\xymatrix{
NQ\sqcup _{NP}NR \ar[d]_{\bar{\eta } } \ar[dr]^\eta \ar[r]^(.4){\bar{\eta } } & NQ\sqcup _{NW}D(NW\sqcup _{NP}NR) \ar[d]^\xi \\
NQ\sqcup _{NW}D(NW\sqcup _{NP}NR) & D(NQ\sqcup _{NP}NR) \ar@{-->}[l]
}
\end{displaymath}
that makes it commute. The dashed map precomposed with $\xi$ is the identity as $\eta$ is degreewise surjective, being a cobase change of a degreewise surjective map. Recall that the degreewise surjective maps of $sSet$ are precisely the epimorphisms and, therefore, are right cancellable. This finishes the proof of \cref{lem:proof_of_second_reduction}, which in turn implies that Part 2 of \cref{lem:second_reduction} holds.

