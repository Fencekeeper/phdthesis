

\section{Calculations of some cofibrant small categories}
\label{sec:calculations}



We will prove the following.
\begin{lemma}\label{lem:q_of_barratnerve_of_subdivision}
Let $X$ be a simplicial set. The map
\[q(t_X):qDSd\, X\to qB\, X\]
is an isomorphism.
\end{lemma}
\noindent The map $q(t_X)$ is what we get by applying $p:Cat\to PoSet$ to the map in the lower right part of the triangle
\begin{displaymath}
\xymatrix{
cSd\, X\ar[dd]_{c(b_X)} \ar[dr]^{c(\eta _{Sd\, X})} \\
& cUDSd\, X \ar[ld]^{c(Ut_X)} \\
cUB\, X
}
\end{displaymath}
as $q=pcU$.

To prove \cref{lem:q_of_barratnerve_of_subdivision} it is more or less enough to know how to define $c:sSet\to Cat$. For the reader's convenience we recall that construction now.

Let $X$ be a simplicial set. Take the (directed) graph $G=(O,A)$ whose objects are
the $0$-simplices $O=X_0$ and whose arrows are the $1$-simplices $A=X_1$. The vertex operators $\varepsilon _j:[0]\to [1]$, $j=0,1$, define the source and target functions $\varepsilon _0^*,\varepsilon _1^*:A\to O$, respectively. The morphisms $o\to o'$ of the free category $C(G)$ generated by the graph $G$ are the finite strings
\[o=x_0\xrightarrow{y_1} x_1\to \dots \xrightarrow{y_n} x_n=o'\]
with $n\geq 0$. Composition is concatenation of strings. The empty strings, meaning strings of length $n=0$, are identities. One defines the categorification of $X$ as the quotient
\[cX=C(G)/\sim\]
under the congruence defined by identifying
\[z\delta _1\sim z\delta _0\circ z\delta _2\]
for $2$-simplices $z$.

If $f:X\to X'$ is a simplicial map, then the induced functor $c(f)$ arises from the diagram
\begin{displaymath}
\xymatrix{
C(G) \ar[d] \ar[r] & C(G') \ar[d] \\
cX \ar@{-->}[r]_{c(f)} & cX'
}
\end{displaymath}
so we can make the observation that $c(f)$ is full if $f_1:X_1\to X'_1$ is surjective, meaning $f$ is surjective in degree $1$. This is of relevance to \cref{lem:q_of_barratnerve_of_subdivision}, which we are now ready to prove.
\begin{proof}[Proof of \cref{lem:q_of_barratnerve_of_subdivision}]
The map
\[t_X=U(t_X):DSd\, X=UDSd\, X\to BX=UB\, X\]
is degreewise surjective as $b_X$ is, so in particular it is surjective in degree $1$. Furthermore, notice that $t_X$ is a bijection in degree $0$. This is because $b_X$ is a bijection in degree $0$ and because $\eta _{Sd\, X}$ is degreewise surjective and in particular surjective in degree $0$.

Now we know that $t_X$ is a bijection in degree $0$



. It follows that $c(Ut_X)$ is a full functor that is a bijection on objects. Its target is a poset, so when applying the posetal reflection to $c(Ut_X)$ its source becomes a poset. In other words, the map becomes an isomorphism.
\end{proof}
\noindent This little discussion has a couple of consequences.

The first consequence is the following.
\begin{lemma}
Let $X$ be a simplicial set. The categorification $cSd^2\, X$ of the two-fold normal subdivision is isomorphic to the poset $Sd(X)^\sharp$ of non-degenerate simplices of the subdivision.
\end{lemma}
\begin{proof}
We apply $U:PoSet\to Cat$ to the isomorphism $pc(b_{Sd\, X})$ and consider the square
\begin{displaymath}
\xymatrix{
cSd^2\, X \ar[d]^\cong _{\eta _{cSd^2\, X}} \ar[r] & cUBSd\, X \ar[d]_\cong ^{\eta _{cUBSd\, X}} \\
UpcSd^2\, X \ar[r]^\cong _{Upc(b_{Sd\, X})} & UpcUBSd\, X
}
\end{displaymath}
in which the lower horizontal map is an isomorphism due to our discussion above. As $cSd^2\, X=U(TX)$ for some poset $TX$, it follows that the instance $\eta _{cSd^2\, X}$ of the unit is an isomorphism. Now, use of the counit
\[cUBSd\, X=cUN(Sd(X)^\sharp )=cNU(Sd(X)^\sharp )\xrightarrow{\cong  } U(Sd(X)^\sharp )\]
finishes the proof of the claim.
\end{proof}
\noindent Supposedly, this isomorphism is known, but the calculation was not pointed out in Thomason's article. Therefore, we do it here.

The second consequence is the analogous one, namely the following.
\begin{corollary}
If \cref{conj:cofibrant_nonsing_simp_set} holds, then $t_{Sd\, X}$ is an isomorphism for any simplicial set $X$.
\end{corollary}
\begin{proof}
The proof is analogous - the only difference is that we specified a comparison map to begin with. One considers the diagram
\begin{displaymath}
\xymatrix{
DSd^2\, X \ar[d]_{\eta _{DSd^2\, X}}^\cong \ar[r] & BSd\, X \ar[d]^{\eta _{BSd\, X}}_\cong \\
NqDSd^2\, X \ar[r]^\cong _{Nq(t_{Sd\, X})} & NqBSd\, X
}
\end{displaymath}
in which the lower horizontal map is an isomorphism due to \cref{lem:q_of_barratnerve_of_subdivision}. Here we let $FX$ be the poset such that
\[DSd^2\, X=N(FX),\]
which exists according to \cref{conj:cofibrant_nonsing_simp_set}.

As $DSd^2\, X$ and $BSd\, X$ are nerves, the instances $\eta _{DSd^2\, X}$ and $\eta _{BSd\, X}$ of the unit are isomorphisms. As $N:PoSet\to nsSet$ is a full embedding it follows that $t_{Sd\, X}$ and its inverse are induced by unique functors $FX\to Sd(X)^\sharp$ and $Sd(X)^\sharp\to FX$. These must be mutual inverses.
\end{proof}

Notice that the congruence generated by $\sim$ makes the degeneracy $x\sigma _0$, $\sigma _0:[1]\to [0]$, of any object $x$ behave as the identity morphism $x\to x$ of $cZ$. One verifies this by checking the following two cases.

The first case is when $y\in Z_1=A$ is an arrow of $G$ with $x=\varepsilon _0^*(y)$. Then the identification above in the case of the $2$-simplex $z=y\sigma _0$ ensures that the triangle
\begin{displaymath}
 \xymatrix{
 & \varepsilon ^*_1(y) \\
 x \ar[ur]^{y=z\delta _1} \ar[rr]_{x\sigma _0=z\delta _2} && x=\varepsilon ^*_0(y) \ar[lu]_{y=z\delta _0}
 }
\end{displaymath}
in $cZ$ commutes. The equalities $y=z\delta _1$ and $y=z\delta _0$ are immediate from the fact that $\delta _1$ and $\delta _0$ are sections of $\sigma_0:[2]\to [1]$. The third equality $x\sigma _0=z\delta _2$ comes from the calculation
\[z\delta _2=(y\sigma _0)\delta _2=y(\sigma _0\delta _2)=y(\varepsilon _0\sigma _0)=(y\varepsilon _0)\sigma _0=x\sigma _0.\]
In other words, the string
\[x=\varepsilon ^*_0(y)\xrightarrow{y} \varepsilon ^*_1(y)\]
of length $1$ is identified with the string
\[x\xrightarrow{x\sigma _0} x=\varepsilon ^*_0(y)\xrightarrow{y} \varepsilon ^*_1(y)\]
of length $2$. If $y=f_1$ for a morphism
\[y_0\xrightarrow{f_1} y_1\to \dots \xrightarrow{f_n} y_n\]
of $C(G)$ denoted $f$, say of length $n$, then the concatenation $\langle x\sigma _0,f\rangle$ of $x\sigma _0$
and $f$ is identified with $f$. This is
because $f$ is the concatenation of $f_1$ and the string $y_1\xrightarrow{f_2} y_1\to \dots \xrightarrow{f_n} y_n$ of
length $n-1$.

The second case is when $y\in Z_1=A$ is an arrow of $G$ with $x=\varepsilon _1^*(y)$. In this case we let $z=y\sigma _1$, $\sigma _1:[2]\to [1]$. An analogous argument shows that the string
\[\varepsilon ^*_0(y)\xrightarrow{y=z\delta _1} x=\varepsilon ^*_1(y)\]
of length $1$ is identified with the string
\[\varepsilon ^*_0(y)\xrightarrow{y=z\delta _1} \varepsilon ^*_1(y)=x\xrightarrow{x\sigma _0} x\]
of length $2$.

This means that we can conclude that $x\sigma _0$ behaves as the identity $x\to x$. Such behavior uniquely determines a morphism, so the empty string
\[x,\]
which is the identity $x\to x$ of the free category $C(G)$, must be identified with the string $x\xrightarrow{x\sigma _0} x$ under the congruence generated by $\sim$ and it can therefore be regarded as (a representative for) an identity morphism of $cZ$.




Because $c$ takes degreewise surjective maps to full functors, the functors $c(Ut_{Sd\, X})$ and $c(\eta _{Sd^2\, X})$ are full. These are also bijections on the objects. Thus it will follow that $q(Ut_{Sd\, X})$ is an isomorphism if $pc(b_{Sd\, X})$ is. Hence, we should study the map $pc(b_{Sd\, X})$. We may as well replace $Sd\, X$ with an arbitrary simplicial set $Y$ in the following discussion.



We are only concerned with those categories whose objects are precisely the non-degenerate simplices. Posetal reflections of such simplex categories are simply quotient categories.

The categories $cSd\, Y$ and $cUB\, Y$ can be viewed as (different) simplex categories of $Y$. In both cases, the set of objects is the set of non-degenerate simplices. The counit
\[cUB\, Y=cUN(Y^\sharp )=cNU(Y^\sharp )\xrightarrow{\cong} U(Y^\sharp )\]
of the adjunction
\[c:sSet\rightleftarrows Cat:N\]
identifies $cUB\, Y$ with $Y^\sharp$. In $Y^\sharp$, there is a morphism $z\to y$ in $Y^\sharp$ if and only if $z$ is a face of $y$. A non-degenerate simplex can be a face of another non-degenerate simplex in different ways, so $(-)^\sharp$ can be said to be the most drastic method of forming a simplex category. As $cSd\, Y$ has an awkward description it would be beneficial to compare it to a bigger simplex category with an elementary description. Towards the goal of finding such a category we contemplate the subdivision $Sd\, Y$.

The simplex category $cSd\, Y$ of $Y$ is defined in the following way. One takes the free category $C(G)$ of the directed graph $G=(Sd(Y)_0,Sd(Y)_1)$ whose nodes are the $0$-simplices of $Sd\, Y$ and whose edges are the $1$-simplices going from the zeroeth to the first vertex. Then one introduces an appropriate congruence that comes from the wish to consider the first face of a $2$-simplex as being equal to the zeroeth face composed with the second face.

Generating the morphisms of $C(G)$, then, are the $1$-simplices $[y,\varphi ]$ of $Sd\, Y$, and they are uniquely represented by minimal pairs
\[((y,\varphi =(\varphi _0,\varphi _1)).\]
This means that $y$ is non-degenerate and that $\varphi _1$ is the identity $\iota :[n_y]\to [n_y]$ where $n_y$ is the degree of $y$. Here, $\varphi _0$ is any face operator $\mu :[k]\to [n_y]$. The zeroeth vertex $[y,\varphi ]\varepsilon _0$ is represented by
\[(y,(\varphi _0))=(y,(\mu \iota _{[k]}))=(y,((\mu \iota _{[k]})^\sharp ))\sim (y\mu ,(\iota _{[k]}))\sim ((y\mu )^\sharp ,(y\mu )^\flat (\iota _{[k]})).\]
Actually, after having unravelled the generators of the morphisms of $C(G)$ it begins to become apparent that \cref{lem:q_of_barratnerve_of_subdivision} holds.


