
\chapter{Acknowledgements}

First and foremost I want to thank John who gave me the opportunity to work with this project. It has been very rewarding in and of itself. However, during my time as a PhD candidate I have also had the opportunity to expose myself to many interests that a less independent position would likely be an obstacle to. John is demanding when delivering criticism, but he is also patient and has given me great freedom. He has also encouraged me to give talks and has supervised me in seminars and in self taught topics.

Although I had hoped to explore more of the questions that were posed during my time at the department, it seems that there is after all quite a lot of publishable material at the moment this is written. Moreover, the material has taken shape as four individual papers that form four chapters in this thesis that are formulated in a more independent manner than the rest. John put down a lot of work to help me with the final (?) formulation of the publishable work. Now I feel more than ready to move on to a new period of my life --- an existence that I hope involves some mathematics, although of a slightly more practical nature.

Most of my courses above a certain level were taught by Paul Arne and Bjørn. Paul Arne has a style of teaching that comes with great reward as there is often a lot of effort put into say the problem solving sessions. Bjørn has geometrical insights and experience that I could never extract from any book. The courses they both taught were highly appreciated.

The department administration has been a tremendous help, for example the staff shields the scientific staff from the outside world and is also a bridge to it when it is desired or necessary. Help is immediately available at all times, and links are provided to experts in the faculty staff that can help with say legal matters. It is particularly Yngvar and Biljana that deserve praise in this regard.

The office computer and the software is as stable as I have ever experienced. Any request concerning IT is dealt with superbly and shortly after making contact. If there is ever a problem, then Terje will fix it, handing out new equipment if he has to. I doubt that there is a better supporting staff than the one at the Math. dept.

Karoline was very helpful as she started working in the library, but first and foremost she was a friend and colleague before and after that point in time. Similarly, I shared hallway with Sigurd who has also been a friend and colleague during my time at the department --- one who helped me with the programming of illustrations in LaTeX. Martin provided me with the template in which this is written, which worked excellent and better than anything I used before.

My mother and father have been supportive in everything I ever did --- in all possible aspects. The PhD candidacy is no exception. To have them has been very important, even essential after I established a home and later a family. My daughter gets from them what I myself cannot manage in this hectic and somewhat irregular existence --- and more. Especially my mother deserves thanks as she travels far just to be with her. Concerning my daughter, an even greater role than my mother's is played in daily life by her aunt Nimo. She is like a third parent. Not to mention the fact that she helped us a lot in the period before Sofia was born. It is an understatement to say that she has been there for us beyond the call of duty. I am deeply thankful to her.

More people could certainly have been mentioned here. Furthermore, I have made a few friends for life during my stay at the department. This is yet another by-product of getting the opportunity that the PhD candidacy has been.

