
\chapter{To the reader}

The author’s PhD thesis grew out of the desires, firstly to establish an interesting model structure on non-singular simplicial sets with a connection to simplicial sets that uses the desingularization functor, and secondly to describe desingularization. As a product of this work, the author proved four major results. Each of these theorems developed into a self-contained text that the author intends to try to publish. Therefore, this dissertation is not quite a monograph, but rather four attemped articles with connections between them and with a surrounding text that supports them so as to supplement the material and make all of the material readable in a suitable context. The four articles are the chapters \ref{ch:it_desing},\ref{ch:exponentials},\ref{ch:htythy} and \ref{ch:optriang}. Consequently, there may be advantages and disadvantages for the reader, compared with reading a monograph. 

Presumably, there are two possible minimum requirements for reading this dissertation. A master student with some prior knowledge of simplicial sets and of model categories should be able to read the dissertation. So too, should a more experienced topologist that is familiar with homotopy theory, but without prior knowledge of model categories. \cref{ch:model} is a minimal treatment of model categories. For a master student with prior knowledge, this chapter serves to fix notation and terminology. On the other hand, a more experienced topologist that is familiar with homotopy theory, is likely to surmise the motivation behind model structures by reading \cref{ch:model}. 

One possible disadvantage with reading this hybrid between a monograph and an article-based dissertation is that some material is repeated, in order to make the four chapters \ref{ch:it_desing},\ref{ch:exponentials},\ref{ch:htythy} and \ref{ch:optriang} self-contained. 

A possible advantage is the opportunity to read any of the three articles in chapters \ref{ch:it_desing},\ref{ch:exponentials} or \ref{ch:optriang} without looking at the other material prior to the reading. The reason that this is possible is that the number of references out of the four attempted articles is kept to a minimum. As such, it is realistic to immediately read any one of the chapters \ref{ch:it_desing},\ref{ch:exponentials} or \ref{ch:optriang} with the occational glance at the few external references.

However, to read \cref{ch:htythy}, one might want read \cref{ch:model} first, because \cref{ch:htythy} concerns the establishing of a model structure on non-singular simplicial sets, and \cref{ch:model} is our minimal treatment of model categories. In the event that the reader is already familiar with Hirschhorn's or Hovey's book, \cref{ch:model} can be skipped and then looking up the occational reference if needed.

Another possibility is to just read the chapters in order. This is likely to be more rewarding. \cref{ch:htythy} builds on \cref{ch:it_desing} and \cref{ch:exponentials}, whereas \cref{ch:optriang} builds on \cref{ch:htythy}.

Reading any of the chapters in the complement of the chapters \ref{ch:it_desing},\ref{ch:exponentials},\ref{ch:htythy} and \ref{ch:optriang} out of context is not particularly meaningful, as the complement is simply there to supplement the four attempted articles.

