
\section{Strategy to establish the model structure}
\label{sec:structure}

\noindent We find ourselves in a similar situation as that of Thomason. Prior to his article \cite{Th80} there was a homotopy theory of small categories for which Quillen's paper \cite{Qu67} is a reference. It is thought of as inherited from topological spaces via the classifying space. The nerve induces an equivalence of the homotopy categories, yet its left adjoint $c:sSet\to Cat$ does not induce an (inverse) equivalence.

After the recent development of his time, Thomason discovered that the geometrically favorable construction $cSd^2$ preserves homotopy type \cite{Th80} and managed to put a model structure on small categories that makes it Quillen equivalent to simplicial sets, with $cSd^2$ as the left Quillen functor. Fritsch and Latch \cite{FL81} present a contemporary view of the historical development and explain how surprising the result was.

Similarly, there exists a homotopy theory of ordered simplicial complexes thought of as inherited from simplicial sets. The category of ordered simplicial complexes is slightly smaller than $nsSet$. The inclusion $U:nsSet\to sSet$ is full by definition and has a left adjoint called desingularization, as we explained in \cref{sec:intro_hty}. We will display examples of the behavior of desingularization in \cref{sec:behavior}.

There are two main differences between our situation and that of Thomason, namely that we can build on his work and that desingularization is in some sense more difficult to work with.

Categorification $c:sSet\to Cat$ has the following rather elementary description. For $X$ a simplicial set, let the the set of objects $obj(cX)$ of $cX$ be the set $X_0$ of $0$-simplices. The morphisms are freely generated by the set $X_1$ of $1$-simplices with $x\in X_1$ viewed as a morphism $x\delta _1\to x\delta _0$, and then imposing a composition relation $x\delta _1=x\delta _0\circ x\delta _2$, for all $2$-simplices $x\in X_2$. Here, $\delta _j$ is the elementary face operator that omits the index $j$.

On the other hand, desingularization has the two descriptions given in \cref{def:desing} and \cref{thm:main_result_itdesing}. In general, these can be more difficult to work with. We will essentially be using the latter description, albeit a modification.

The strategy we shall use to obtain the model structure on $nsSet$ is essentially the lifting method that Thomason \cite{Th80} uses, except that it has become standardized. It is summarized in the following theorem, credited to D. M. Kan. The language we use is that of Theorem 11.3.2 in Hirschhorn's textbook \cite[p.~214]{Hi03}.
\begin{theorem}[D.M. Kan]\label{thm:lifting_across_adjunction}
Suppose there is an adjunction
\[F:\mathscr{M} \rightleftarrows \mathscr{N} :G\]
where $\mathscr{M}$ is a cofibrantly generated model category with $I$ as the set of generating cofibrations and $J$ as the set of generating trivial cofibrations. Furthermore, assume that $\mathscr{N}$ is a bicomplete category. If
 \begin{enumerate}
  \item {(First lifting condition) each of the sets $FI$ and $FJ$ permits the small object argument, and}
  \item{(Second lifting condition) $G$ takes relative $FJ$-cell complexes to weak equivalences,}
 \end{enumerate}
then $\mathscr{N}$ is a cofibrantly generated model category where the weak equivalences of $\mathscr{N}$ are the morphisms $f$ such that $Gf$ is a weak equivalences, and where $FI$ and $FJ$ are the generating cofibrations and generating trivial cofibrations, respectively. Moreover, $(F,G)$ becomes a Quillen pair.
\end{theorem}
\noindent Formalities ensure that a morphism $f$ in $\mathscr{N}$ is a fibration in the lifted model structure if and only if $Gf$ is a fibration. The language of \cref{thm:lifting_across_adjunction} is fairly standard, but it will be interpreted or explained to a suitable extent when we get to the relevant part.

We will make use of \cref{thm:lifting_across_adjunction} in order to establish the model structure by considering the case when
\[(F,G)=(DSd^2,Ex^2U)\]
and when $sSet$ has the standard model structure.

Recall \cref{not:prototypes_cofibr_trivial_cofibr_standard_pre_exist}. In our case, $I$ serves as a set of generating cofibrations for $sSet$ and $J$ serves as a set of generating trivial cofibrations for $sSet$. The method of lifting the standard model structure on $sSet$ to $nsSet$ is justified by the fact that $U(f)$ is a weak equivalence if and only if $Ex^2U(f)$ is a weak equivalence.

The key to verifying the second lifting condition is the notion Str\o m map, introduced in \cref{def:strom}. Str\o m maps have good technical properties, as shown by \cref{prop:Strom-maps_closed_under_cobasechange}, and good homotopical properties, as shown by \cref{lem:Pushout_along_strom_homotopically_wellbehaved}. At the same time, the class of Str\o m maps contains the sets $DSd^2(I)$ and $DSd^2(J)$, which \cref{cor:two-fold_subdivision_strom} shows.




