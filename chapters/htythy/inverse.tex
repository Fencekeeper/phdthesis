
\section{A homotopy inverse of the inclusion}
\label{sec:inverse}

In this section, we prove that the Quillen pair $(DSd^2,Ex^2U)$ is indeed a Quillen equivalence. This is stated as \cref{prop:homotopy_inverse} below. In other words, towards the end of this section, we have sufficient knowledge to establish \cref{thm:main_homotopy_theory}, which is our main result.

Intuitively, the first step towards establishing the Quillen equivalence is the following result.
\begin{proposition}\label{prop:desing_double_subdvision_homotopy_inverse}
Let $X$ be a simplicial set. The unit $Sd^2\, X\to UDSd^2\, X$ of the adjunction
\begin{displaymath}
 \xymatrix{
 sSet \ar@<+.7ex>[r]^D & nsSet \ar@<+.7ex>[l]^U
 }
\end{displaymath}
is a weak equivalence.
\end{proposition}
\begin{proof}
Consider the skeleton filtration
\[X^0\to X^1\to \cdots \to X^n\to \cdots \]
of $X$, given by successively attaching the non-degenerate $k$-simplices to the $(k-1)$-skeleton, $k>0$. Note that $Sd^2\, X^n$ can be built from $Sd^2\, X^{n-1}$ as the Kan subdivision preserves colimits and degreewise injective maps \cite[Prop.~4.6.3~(i),~p.~200]{FP90}.

By naturality, the unit $Sd^2\, X\to UDSd^2\, X$ arises as a map between sequential colimits from the diagram
\begin{displaymath}
\xymatrix{
Sd^2\, X^0 \ar[d]^\cong \ar[r] & Sd^2\, X^1 \ar[d] \ar[r] & \cdots \ar[r] & Sd^2\, X^n \ar[d] \ar[r] & \cdots \\
UDSd^2\, X^0 \ar[r] & UDSd^2\, X^1 \ar[r] & \cdots \ar[r] & UDSd^2\, X^n \ar[r] & \cdots
}
\end{displaymath}
in $sSet$. This is because $D$ is a left adjoint and because $U:nsSet\to sSet$ preserves filtered colimits by \cref{lem:filtered_colimits_preservation}.

If $Sd^2\, X^n\to UDSd^2\, X^n$ is a weak equivalence for each $n\geq 0$, then $Sd^2\, X\to UDSd^2\, X$ is a weak equivalence. Now, the map
\[Sd^2\, X^0\xrightarrow{\cong} UDSd^2\, X^0\]
is an isomorphism for every $X$, because every $0$-dimensional simplicial set is non-singular.

Suppose $n>0$ is such that $Sd^2\, X^{n-1}\to UDSd^2\, X^{n-1}$ is a weak equivalence.  Hence, the diagram
\begin{equation}
\label{eq:diagram_proof_of_prop_desing_double_subdvision_homotopy_inverse}
\begin{gathered}
\xymatrix@=1em{
Sd^2(\bigsqcup _{x\in X^\sharp _n}\Delta [n]) \ar[d]^\cong & Sd^2(\bigsqcup _{x\in X^\sharp _n}\partial \Delta [n]) \ar[l] \ar[d]^\cong \ar[r] & Sd^2\, X^{n-1} \ar[d]^\sim \\
UDSd^2(\bigsqcup _{x\in X^\sharp _n}\Delta [n]) & UDSd^2(\bigsqcup _{x\in X^\sharp _n}\partial \Delta [n]) \ar[l] \ar[r] & UDSd^2\, X^{n-1}
}
\end{gathered}
\end{equation}
in $sSet$ yields a factorization
\[Sd^2\, X^n\xrightarrow{\sim } Z\to UDSd^2\, X^n\]
of the unit $Sd^2\, X^n\to UDSd^2\, X^n$ as a map between the pushouts $Sd^2\, X^n$ and $Z$ in $sSet$ followed by a canonical map $Z\to UDSd^2\, X^n$.

By the glueing lemma, the map $Sd^2\, X^n\xrightarrow{\sim } Z$ is a weak equivalence as the two left hand horizontal maps of (\ref{eq:diagram_proof_of_prop_desing_double_subdvision_homotopy_inverse}) are degreewise injective.

The map
\[Sd^2(\bigsqcup _{x\in X^\sharp _n}\partial \Delta [n])\to Sd^2(\bigsqcup _{x\in X^\sharp _n}\Delta [n])\]
is a Str\o m map by \cref{cor:two-fold_subdivision_strom}. By \cref{lem:Pushout_along_strom_homotopically_wellbehaved} it therefore follows that $Z\xrightarrow{\sim } UDSd^2\, X^n$ is a weak equivalence.
\end{proof}
\noindent Thus we obtain the fact that the homotopy type is preserved when we apply desingularization to the double Kan subdivision of some simplicial set.

Our second step is to move from considering the adjunction $(D,U)$ to considering the adjunction $(DSd^2,Ex^2U)$.
\begin{lemma}\label{lem:unit_weak_eq}
The unit $\eta _X:X\to Ex^2UDSd^2X$ is in general a weak equivalence.
\end{lemma}
\noindent \cref{lem:unit_weak_eq} will follow from the bulk of the proof of \cref{prop:second_condition_lifting_criterion}. In the language of Fritsch and Latch \cite{FL81}, the construction $DSd^2$ is a homotopy inverse for the inclusion $U:nsSet\to sSet$.
\begin{proof}[Proof of \cref{lem:unit_weak_eq}.]
The unit of $(DSd^2,Ex^2U)$ is that of the composite adjunction
\begin{displaymath}
 \xymatrix{
 sSet \ar@<+.7ex>[r]^{Sd^2} & sSet \ar@<+.7ex>[l]^{Ex^2} \ar@<+.7ex>[r]^D & nsSet \ar@<+.7ex>[l]^U
 }
\end{displaymath}
and is therefore itself the composite
\begin{equation}
\label{eq:diagram_proof_of_lem_unit_weak_eq}
\begin{gathered}
\xymatrix{
X \ar[r]^(.3)\sim & Ex^2(Sd^2\, X) \ar[r] & Ex^2(UD(Sd^2\, X))
}
\end{gathered}
\end{equation}
where the first map is known to be a weak equivalence. To see that the latter statement is true, it is enough to realize that the unit $X\to ExSd\, X$ of $(Sd,Ex)$ is a weak equivalence.

Adjoint \cite[p.~213]{FP90} to the last vertex map $d_X:Sd\, X\xrightarrow{\sim } X$ is a natural weak equivalence $e_X:X\xrightarrow{\sim } Ex\, X$ \cite[Lem.~4.6.20]{FP90}. The unit of $(Sd,Ex)$ is adjoint to the identity $Sd\, X\to Sd\, X$. Moreover, the unit of $(Sd,Ex)$ fits into the commutative triangle
\begin{displaymath}
\xymatrix{
& ExSd\, X \ar[dr]^{Ex(d_X)}_\sim \\
X \ar[ur] \ar[rr]_{e_X}^\sim && Ex\, X
}
\end{displaymath}
as we see from the commutative square
\begin{displaymath}
\xymatrix{
sSet(Sd\, X,Sd\, X) \ar[d]_{sSet(id,d_X)} \ar[r]^\cong & sSet(X,ExSd\, X) \ar[d]^{sSet(id,Ex(d_X))} \\
sSet(Sd\, X,X) \ar[r]_\cong & sSet(X,Ex\, X)
}
\end{displaymath}
in which $d_X$ is sent to $e_X$ under the lower horizontal map by definition and in which the identity is sent to the unit of $(Sd,Ex)$ under the upper horizontal map. The latter square implies that $e_X$ can be obtained by postcomposing the unit with $Ex(d_X)$. The two-out-of-three property implies that the unit is a weak equivalence.

The second map of the composite (\ref{eq:diagram_proof_of_lem_unit_weak_eq}) is the result of applying $Ex^2$ to the unit
\[Sd^2\, X\xrightarrow{\sim } UDSd^2\, X,\]
which is a weak equivalence by \cref{prop:desing_double_subdvision_homotopy_inverse}. Now, the functor $Ex^2$ preserves weak equivalences. This shows that the composite (\ref{eq:diagram_proof_of_lem_unit_weak_eq}) is a weak equivalence.
\end{proof}
\noindent Having proven that the unit of the Quillen pair $(DSd^2,Ex^2U)$ is a weak equivalence is in fact enough, in our case, to prove that the Quillen pair is indeed a Quillen equivalence.

We have so far followed Hirschhorn's terminology throughout this article. However, to prove \cref{prop:homotopy_inverse}, we will use a result in Hovey's book. Hirschhorn's and Hovey's definitions of the term Quillen equivalence are identical to the following.
\begin{definition}
Suppose $F:\mathscr{M} \rightleftarrows \mathscr{N} :G$ a Quillen pair with
\[\varphi :\mathscr{N} (FX,Y)\xrightarrow{\cong } \mathscr{M} (X,GY)\]
the natural bijection that comes with the underlying adjunction $(F,G)$ of categories. We say that $(F,G)$ is a \textbf{Quillen equivalence} if $f:FX\to Y$ is a weak equivalence in $\mathscr{N}$ if and only if $\varphi (f):X\to GY$ is a weak equivalence in $\mathscr{M}$ whenever $X$ is a cofibrant object of $\mathscr{M}$ and $Y$ is a fibrant object of $\mathscr{N}$.
\end{definition}
\noindent Moreover, this definition is independent of any choice of functorial factorizations and any choice of fibrant and cofibrant replacement functors.

A canonical choice of fibrant and cofibrant replacement functors are implicitly part of the model structure in Hovey's notion of model category \cite[Def.~1.1.3, p.~3]{Ho99}, whereas the opposite is true in Hirschhorn's notion \cite[Def.~7.1.3, p.~109]{Hi03}. Namely, Hirschhorn assumes the existence of two functorial factorizations, one as a cofibration followed by a trivial fibration and another as a trivial cofibration followed by a fibration. However, Hovey makes such a choice of functorial factorizations part of the model structure. Thus arises canonical fibrant and cofibrant replacement functors. To think of $(DSd^2,Ex^2U)$ as a Quillen pair according to Hovey, we must then make a choice of functorial factorizations for each of the model categories $sSet$ and $nsSet$.

Now, \cref{thm:lifting_across_adjunction} is the lifting theorem \cite[Thm.~11.3.2]{Hi03}, which applies the recognition theorem \cite[Thm.~11.3.1]{Hi03} whose proof uses the small object argument in the form \cite[Prop.~10.5.16]{Hi03}. From the latter result, which is more or less a standard formulation, we can read off that the small object argument establishes two functorial factorizations on $nsSet$, one into a relative $DSd^2(I)$-cell complex followed by a $DSd^2(I)$-injective, and another into a relative $DSd^2(J)$-cell complex followed by a $DSd^2(J)$-injective. We choose these to serve as part of the model structure on $nsSet$ according to Hovey's notion. Clearly, we follow the same procedure with regards to the sets $I$ and $J$ of maps in $sSet$.

When choices of functorial factorizations have been made, there is a canonical fibrant replacement functor $R$ in $nsSet$ that arises from the factorization
\[A\xrightarrow{r_A} RA\to \Delta [0]\]
of the terminal map, for each non-singular $A$, as a relative $DSd^2(J)$-cell complex $r_A$ followed by a fibration $RA\to \Delta [0]$. In other words, the non-singular simplicial set $A$ is replaced by a fibrant non-singular simplicial set $RA$, with a natural map $r_A$ from the original to its replacement.

The choices of functorial factorizations can simply be forgotten after the proof of \cref{prop:homotopy_inverse}. Because the term Quillen equivalence is defined the same way by both Hirschhorn and Hovey and because this definition has no reference to fibrant or cofibrant replacements, the pair $(DSd^2,Ex^2)$ will be a Quillen equivalence according to Hirschhorn if it is according to Hovey.

Finally, we obtain the last piece used to establish \cref{thm:main_homotopy_theory}, which is the main result.
\begin{proposition}\label{prop:homotopy_inverse}
The Quillen pair
\[DSd^2:sSet\rightleftarrows nsSet:Ex^2U\]
is a Quillen equivalence.
\end{proposition}
\begin{proof}
The pair $(DSd^2,Ex^2U)$ is a Quillen equivalence \cite[Cor.~1.3.16]{Ho99} if and only if $Ex^2U$ reflects weak equivalences between fibrant objects and the composite
\[X\xrightarrow{\eta _X} Ex^2UDSd^2X\xrightarrow{Ex^2U(r_{DSd^2X})} Ex^2URDSd^2X\]
is a weak equivalence for every cofibrant $X$. Here,
\[r_{DSd^2\, X}:DSd^2\, X\xrightarrow{\sim } RDSd^2\, X\]
is the natural relative $DSd^2(J)$-cell complex that comes with the fibrant replacement $R$.

As the model structure on $nsSet$ is lifted along the right adjoint $Ex^2U$, this functor  reflects weak equivalences without an assumption on either the source or the target. For the same reason, the functor $Ex^2U$ preserves weak equivalences. Any object in $sSet$ is cofibrant. Nevertheless, it follows that \cref{prop:homotopy_inverse} holds if the following result holds, which it does.
\end{proof}
\begin{proof}[Proof of \cref{thm:main_homotopy_theory}.]
First, by \cref{prop:main_homotopy_theory}, the category $nsSet$ is a cofibrantly generated model category and $(DSd^2,Ex^2U)$ is a Quillen pair when $sSet$ is equipped with the standard model structure due to Quillen. Second, the model category $nsSet$ satisfies the axiom of propriety according to \cref{prop:axiom_of_propriety}. Finally, \cref{prop:homotopy_inverse} says that the pair $(DSd^2,Ex^2U)$ is a Quillen equivalence.
\end{proof}




