\chapter{Declaration}

\noindent
To the best of the PhD candidate's knowledge, the following achievement is my own. Any work that is not mentioned here in this declaration is likely not original, or in other words someone else's. In that case, said work is just part of this written presentation to make the story as self-contained as possible. I trust that I have held a reasonable account of references.

There are three results that are my own.

My first result is that one can talk about the simplicial mapping set in the category of non-singular simplicial sets. The need for such a result came up during the work to establish my second result. Rather, it is sufficient to be able to talk about simplicial path sets. This special case is by far the most work intensive part of the proof. My supervisor, John Rognes, had at an earlier time asked himself whether simplicial mapping sets make sense in the category of non-singular simplicial sets, though with another application in mind.

My second result is that the standard model structure on simplicial sets that is due to Quillen, in which the weak equivalences are those that induce weak homotopy equivalences and the fibrations are the Kan fibrations, can be lifted across the inclusion followed by double extension, which is right adjoint to twice Kan's subdivision. The style of lifting is a standardized method due to Kan. Largely, the ideas that make the lifting possible come from theory of regular neighborhoods. The choice of Kan's double subdivision followed by desingularization as the homotopy inverse of the inclusion is of course barrowed by Thomason \cite{Th80}. I also barrow his idea of using an auxiliary class of morphisms to aid in the lifting. His term Dwyer morphism was a source of inspiration in that regard, but even more so were Cisinski's correction \cite{Ci99} of Thomason's mistake and a characterization of cofibrations in topological spaces due to Strøm \cite{St66}.

My third result is that Kan's subdivision performed twice followed by desingularization and Kan's subdivision followed by the Barratt nerve are one and the same (up to a natural isomorphism). Waldhausen, Jahren and Rognes (my supervisor) believed in this statement, but used the latter construction rather than the former as a tool in their work \cite{WJR13} as they did not have a proof. Thomason's term Dwyer map \cite{Th80} was useful in one of the steps I took to reduce the problem to its core. Theory of mapping cylinders is very close to this as it is a special case, but I feel that the term Dwyer map most appropriately captures the relevant structures in that step of the problem reduction. In said step I also use a result by Thomason and even a technique from a proof of another result, both from the same article \cite{Th80}.

Furthermore, the reader may notice that I have provided an alternative description of desingularization. This is not quite original as it resembles how Lewis \cite[pp.~158]{Le78} made a space Hausdorff, essentially by identifying two points until a quotient with the Hausdorff property emerges. However, I did more or less make my description of desingularization before my advisor told me of the work of Lewis. When I did learn of his construction I decided to match the method and notation with his, to a certain extent, for comparison.