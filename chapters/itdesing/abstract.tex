

\begin{abstract}
\noindent A simplicial set is said to be \textbf{non-singular} if the representing map of each non-degenerate simplex is degreewise injective. The inclusion into the category of simplicial sets, of the full subcategory whose objects are the non-singular simplicial sets, admits a left adjoint functor called desingularization. In this paper, we provide an iterative description of desingularization that is useful for theoretical purposes as well as for doing calculations.
\end{abstract}

