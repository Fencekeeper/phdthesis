

\section{Calculational methods}
\label{sec:calculations}

As far as we know, the only explicit description of $DX$ that is present in the literature is that of \cref{def:desingularization}. It has the advantage that we easily obtain \cref{lem:desing_unique_factorization_through_unit}. However, the description is otherwise rather difficult to work with. Consequently, we would like some tools to aid in calculation. In this section, we will make a couple of observations that are actually enough to perform a few simple, yet interesting desingularizations.

It is maybe in order that the following near-trivial example be mentioned first.
\begin{example}\label{ex:simplest_non-trivial_desingularization}
Consider a simplicial set $X$ whose set $X_0$ of $0$-simplices is a singleton. It follows immediately from the definition of the term non-singular that any simplex of $A=DX$ is degenerate if its degree is $1$ or higher. If $a$ is a simplex of $A$, then we can write it uniquely as a degeneration
\[a=a^\sharp a^\flat\]
of a non-degenerate simplex $a^\sharp$, by the Eilenberg-Zilber lemma. As we have just argued, the only non-degenerate simplex is the single $0$-simplex, so $a^\sharp$ is that one. If $a$ and $b$ have the same dimension, $n$ say, then
\[a^\flat =a^\flat\]
as there is only one operator $[n]\to [0]$. This proves that the set $A_n$ of $n$-simplices is a singleton, implying that the unique map
\[DX\xrightarrow{\cong } \Delta [0]\]
is an isomorphism.
\end{example}
\noindent Arguably, \cref{ex:simplest_non-trivial_desingularization} is the simplest non-trivial example.

Let $X$ be a simplicial set. Towards the goal of making it non-singular we would need to force any non-embedded non-degenerate simplex into becoming degenerate. Suppose $x\in X^\sharp _{n_x}$. The simplex $x$ is embedded if and only if its vertices are pairwise distinct. If it is not embedded, then we would like to make it degenerate according to any pairwise equalities between its vertices. To achieve this we begin by defining a reflexive, symmetric binary relation $\sim$ on
\[O([n_x])=\{ 0,\dots ,n_x\}\]
by letting
\[i\sim j\Leftrightarrow x\varepsilon _i=x\varepsilon _j.\]
Next, we can define a reflexive binary relation $\approx$ on $O([n_x])$ by letting $i\approx k$ if and only if there is a $j$ such that $i\leq k\leq j$ in the total order on $[n_x]$ and such that $i\sim j$. If $i\sim j$ and $i\leq j$, then $i\approx j$. This means that $\sim$ is contained in the equivalence relation $\simeq$ on $O([n_x])$ that is generated by $\approx$.

Crucially, the equivalence relation $\simeq$ has the property described in the following result.
\begin{lemma}\label{lem:equivalence_relation_gives_rise_to_enforcer}
The equivalence relation $\simeq$ on $O([n_x])$ that is generated by $\approx$ has the property that if $i\simeq j$ and if $i\leq k\leq j$ in the total order on $[n_x]$, then $i\simeq k$.
\end{lemma}
\begin{proof}
Assume that $i\simeq j$ and that $i\leq k\leq j$ in the total order on $[n_x]$. Consider the non-trivial case $i<j$.

In the special case when $i\approx j$, there is a $j'$ such that $i\leq j\leq j'$ and $i\sim j'$. As $j\leq j'$ and $i\leq k\leq j$ we get that $i\leq k\leq j'$. Because $i\sim j'$ we then get that $i\approx k$ from the definition of this binary relation, which implies $i\simeq k$.

If it is not true that $i\approx j$, then we still have elements
\[i_0,\dots ,i_q\in O([n_x])\]
for some $q>1$ such that
\begin{displaymath}
\begin{array}{rcl}
i & = & i_0 \\
i_q & = & j
\end{array}
\end{displaymath}
and
\begin{displaymath}
\begin{array}{rclcrcl}
i_0 & \approx & i_1 & \textrm{ or } & i_1 & \approx & i_0 \\
&&& \dots \\
i_{q-1} & \approx & i_q & \textrm{ or } & i_q & \approx & i_{q-1}
\end{array}
\end{displaymath}
There is some $p<q$ such that $i_p\leq k\leq i_{p+1}$, in the case when $i_p\approx i_{p+1}$, or that $i_p\geq k\geq i_{p+1}$, in the case when $i_{p+1}\approx i_p$. Thus $i\simeq k$.
\end{proof}
\noindent An immediate consequence of \cref{lem:equivalence_relation_gives_rise_to_enforcer} is that the set $O([n_x])/\simeq$ has a canonical total order $\leq$ that the canonical function
\[O([n_x])\to O([n_x])/\simeq\]
respects.

If $m_x+1$ is the cardinality of the set $O([n_x])/\simeq$, then the canonical identification
\[(O([n_x])/\simeq ,\leq )\xrightarrow{\cong } [m_x]\]
suggested above gives rise to the method of enforcing the rules of glueing in $nsSet$.
\begin{definition}\label{def:enforcer}
Let $x$ be a non-degenerate simplex of some simplicial set. Define $\rho _x$ as the composite
\[[n_x]\to (O([n_x])/\simeq ,\leq )\xrightarrow{\cong } [m_x].\]
Let the degeneracy operator $\rho _x$ be known as the \textbf{enforcer of $x$}.
\end{definition}
\noindent In general, the degeneracy operators whose source is $[n_x]$ correspond to equivalence relations on the set $O([n_x])$ that satisfy precisely the condition from \cref{lem:equivalence_relation_gives_rise_to_enforcer}.

The name of $\rho _x$ is meant to signify that it has a role in making sure that the result of desingularizing $X$ is a simplicial set that obeys the rules of glueing in the category $nsSet$. These are stricter than the rules in the category $sSet$. By construction, the enforcer deals with any equalities between vertices of $x$, but in the least drastic manner. It is proper if and only if $x$ is not embedded.
\begin{proposition}\label{prop:role_of_enforcers}
Let $J\subseteq X^\sharp$ be some set of non-degenerate simplices. There is a canonical map
\[\bigsqcup _{j\in J}\Delta [m_j]\to UDX\]
such that the square
\begin{equation}
\label{eq:first_diagram_proof_prop_role_of_enforcers}
\begin{gathered}
\xymatrix{
\bigsqcup _{j\in J}\Delta [n_j] \ar[rr]^{\sqcup _{j\in J}(\rho _j)} \ar[d]_{\vee _{j\in J}(\bar{\jmath } )} && \bigsqcup _{j\in J}\Delta [m_j] \ar[d] \\ 
X \ar[rr]_{\eta _X} && UDX
}
\end{gathered}
\end{equation}
commutes.
\end{proposition}
\begin{proof}
First, note that if $x\in X^\sharp _n$, then the composite
\[\Delta [n_x]\xrightarrow{\bar{x} } X\xrightarrow{\eta _X} UDX\]
factors through $N\rho _x$. One realizes this by considering the image $z$ of $x$ under $\eta _X$. It is uniquely a
degeneracy of a non-degenerate simplex. We get the diagram
\begin{displaymath}
 \xymatrix{
 \Delta [m_x] \ar@{-->}[dr] \\
 & \Delta [m] \ar[dr]^{\overline{z^\sharp } } \\
 \Delta [n_x] \ar[uu]^{\rho _x} \ar[ur]^{z^\flat } \ar[dr]_{\bar{x} } \ar[rr]^{\bar{z} } && UDX \\
 & X \ar[ur]_{\eta _X}
 }
\end{displaymath}
in which $Nz^\flat$ factors uniquely through $N\rho _x$. The explanation for the latter factorization is as follows.

That there is at most one factorization comes from the fact that the nerve $N$ is fully faithful and that $\rho _x$ is epic in $Cat$. That there is a factorization follows from the observation that $z^\flat (i)=z^\flat (i')$ whenever $\rho _x(i)=\rho _x(i')$, as we now argue.

First, suppose $i\sim i'$, meaning $x\varepsilon _i=x\varepsilon _{i'}$. As $\eta _X$ is a simplicial map it follows that
\[z^\sharp \varepsilon _{z^\flat (i)}=z^\sharp z^\flat \varepsilon _i=z\varepsilon _i=z\varepsilon _{i'}=z^\sharp z^\flat \varepsilon _{i'}=z^\sharp \varepsilon _{z^\flat (i')}.\]
The simplicial set $UDX$ is non-singular, so $z^\sharp$ is embedded. Hence,
\[z^\flat (i)=z^\flat (i').\]
Next, as $z^\flat$ is order-preserving we have that $z^\flat (i)=z^\flat (k)$ for each $k$ with $i\leq k\leq i'$. As a consequence, $z^\flat (i)=z^\flat (k)$ whenever $i\approx k$.

In turn, we get that the the equivalence relation $\simeq$ on $O([n_x])$, which corresponds to $\rho _x$, is contained in the equivalence relation that corresponds to $z^\flat$. Thus we obtain a canonical degeneration $w_x$ of $z^\sharp$ such that the square
\begin{displaymath}
\xymatrix{
\Delta [n_x] \ar[d]_{\bar{x} } \ar[r]^{\rho _x} & \Delta [m_x] \ar[d]^{\bar{w} _x} \\
X \ar[r]_{\eta _X} & UDX
}
\end{displaymath}
commutes.

The composites
\[\Delta [n_j]\xrightarrow{\bar{\jmath } } X\xrightarrow{\eta _X} DX,\]
$j\in J$, give rise to a canonical map $\sqcup _{j\in J}\Delta [n_j]\to DX$. The latter can be factored in two different ways due to (\ref{eq:first_diagram_proof_lem_pushout_along_enforcers_intermediate_desingularization}).

The diagram illustrated by
\begin{displaymath}
\xymatrix{
\bigsqcup _{j\in J}\Delta [n_j] \ar@{-->}[r] & \bigsqcup _{j\in J}\Delta [m_j] \ar@{-->}[r] & UDX \\
\Delta [n_x] \ar[u] \ar[r]_{\rho _x} & \Delta [m_x] \ar[u] \ar[ur]_{\bar{w} _x}
}
\end{displaymath}
provides the first of the factorizations that we have in mind and the diagram
\begin{displaymath}
\xymatrix{
\Delta [n_x] \ar[d] \ar[dr]^{\bar{x} } \\
\bigsqcup _{j\in J}\Delta [n_j] \ar@{-->}[r] & X \ar[r]^{\eta _X} & DX
}
\end{displaymath}
provides the second. The promised commutative square consists of precisely these two factorizations.
\end{proof}
\begin{lemma}\label{lem:pushout_along_enforcers_intermediate_desingularization}
Let $X$ be a simplicial set and let $J\subseteq X^\sharp$ be some set of non-degenerate simplices. Consider the cocartesian square
\begin{displaymath}
\xymatrix{
\bigsqcup _{j\in J}\Delta [n_{j}] \ar[rr]^{\sqcup _{j\in J}(\rho _j)} \ar[d]_{\vee _{j\in J}(\bar{\jmath } )} && \bigsqcup _{j\in J}\Delta [m_j] \ar[d] \\
X \ar[rr] && Y
}
\end{displaymath}
in $sSet$. The unit $\eta _X$ factors through the canonical degreewise surjective map $X\to Y$. If $Y$ is non-singular, then the map $Y\to UDX$ of the factorization is an isomorphism.
\end{lemma}
\begin{proof}
As a result of \cref{prop:role_of_enforcers}, the solid diagram
\begin{displaymath}
\xymatrix{
\bigsqcup _{j\in J}\Delta [n_j] \ar[rr]^{\sqcup _{j\in J}(\rho _j)} \ar[d]_{\vee _{j\in J}(\bar{\jmath } )} && \bigsqcup _{j\in J}\Delta [m_j] \ar[d] \ar@/^1pc/[ddr] \\ 
X \ar@/_1pc/[drrr]_{\eta _X} \ar[rr] && Y \ar@{-->}[dr] \\
&&& UDX
}
\end{displaymath}
commutes. Thus a canonical map $Y\to UDX$ arises. It is degreewise surjective as $\eta _X$ is degreewise surjective.

Suppose $Y$ non-singular. We will argue that $Y\to UDX$ is even degreewise injective in this case and that it is thus an isomorphism. We get the commutative diagram
\begin{equation}
\label{eq:first_diagram_proof_lem_pushout_along_enforcers_intermediate_desingularization}
\begin{gathered}
\xymatrix{
& UDX \\
X \ar[dr]_{\eta _X} \ar[ur]^{\eta _X} \ar[rr] && Y \ar[lu] \\
& UDX \ar@{-->}[ur]
}
\end{gathered}
\end{equation}
in which the upper triangle comes from the pushout above and in which the lower triangle comes from $Y$ being non-singular. Hence, the composite
\[UDX\to Y\to UDX\]
is the identity as $\eta _X$ is epic in $sSet$. This implies that $UDX\to Y$ is degreewise injective.

The canonical map $X\to Y$ that comes with the pushout $Y$ is degreewise surjective as it is a cobase change of a degreewise surjective map. Consequently, we can conclude from (\ref{eq:first_diagram_proof_lem_pushout_along_enforcers_intermediate_desingularization}) that the map $UDX\to Y$ is degreewise surjective. This implies that $Y\to UDX$ is degreewise injective in this case.
\end{proof}
\noindent \cref{lem:pushout_along_enforcers_intermediate_desingularization} confirms the intuition that taking the pushout along enforcers is never too drastic.


