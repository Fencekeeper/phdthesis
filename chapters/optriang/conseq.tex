
\section{Applications}
\label{sec:conseq}

In this section, we discuss consequences of \cref{thm:main_opt_triang}.

Interpret $B$ as a functor $sSet\to nsSet$. On the one hand we have the triangulation $BSd:sSet\to nsSet$ of simplicial sets that may seem ad hoc, but that is concrete. On the other hand, we have the functor $DSd^2$ with the same source and target as $BSd$. It is somewhat cryptic as there is no other description of $D$ than the one we gave in \cref{sec:intro}. However, the functor $DSd^2$ has good formal properties. \cref{thm:main_opt_triang} implies that the natural map
\[t_{Sd\, X}:DSd^2\, X\xrightarrow{\cong } BSd\, X\]
is an isomorphism.

The functor $I=BSd$ is already a homotopically good way of making simplicial sets non-singular. It is known from \cite[§2.5]{WJR13} as the \textbf{improvement functor} and plays a role in that book. When we say that the improvement functor is a triangulation, we mean that there is a natural map $UIX\xrightarrow{s_X} X$ whose geometric realization is homotopic to a homeomorphism from the ordered simplicial complex $\lvert UIX\rvert$ to the CW complex $\lvert X\rvert$. The map $s_X$ is particularly well behaved when $X$ is a \textbf{finite simplicial set}, meaning that $X$ is generated by finitely many simplices.

Actually, the functor $DSd^2$ is also a homotopically relevant construction. By \cref{thm:main_homotopy_theory}, it can be made into a left Quillen functor of a Quillen equivalence when $sSet$ is equipped with the standard model structure due to Quillen \cite{Qu67}. Hence, \cref{thm:main_opt_triang} merges two preexisting theories into one.
\begin{definition}\label{def:simple_map_calculating_dsd2}
Let $X$ and $Y$ be finite simplicial sets and let $f:X\to Y$ be a simplicial map. We say that $f$ is \textbf{simple} if the point inverse $\lvert f\rvert ^{-1}(p)$ is contractible for any $p\in \lvert Y\rvert$.
\end{definition}
\noindent The map $s_X$ is simple when $X$ is finite. For a thorough discussion of the construction $I$ and the map $s_X$, see sections $2.2$, $2.3$, $2.5$ and $3.4$ of \cite{WJR13}.

Let $\Delta$ denote the category whose objects are the totally ordered sets $[n]$, $n\geq 0$, and whose morphisms $[m]\to [n]$ are the functions $\alpha$ such that $\alpha (i)\leq \alpha (j)$ whenever $i\leq j$. We refer to the morphisms as \textbf{operators}. Suppose $T:\Delta \to nsSet$ the functor that takes $[n]$ to the barycentric subdivision $\Delta '[n]$ of the standard $n$-simplex. Furthermore, we let $\Upsilon :\Delta \to rsSet$ be the Yoneda embedding $[n]\mapsto \Delta [n]$, corestricted to the full subcategory $rsSet$ of $sSet$ whose objects are the regular simplicial sets. Then $Sd$ is the left Kan extension of $UT$ along $U\Upsilon$.

Two related consequences of \cref{thm:main_opt_triang} are \cref{cor:left_Kan_extension_twice_barycentric_along_Yoneda} and \cref{cor:left_Kan_extension_barycentric_along_Yoneda} below.
\begin{corollary}\label{cor:left_Kan_extension_twice_barycentric_along_Yoneda}
The improvement functor $I:sSet\to nsSet$ is the left Kan extension of $DSdUT$ along $U\Upsilon$.
\end{corollary}
\begin{proof}
Because $Sd$ is the left Kan extension of $UT$ along $U\Upsilon$ and because $DSd$ has a right adjoint, it follows that $DSd^2=DSd\circ Sd$ is the left Kan extension of $DSd\circ UT$ along $U\Upsilon$ \cite[X.5~Thm.~1]{ML98}. The result now follows from \cref{thm:main_opt_triang}.
\end{proof}
\noindent With regards to second corollary, which is \cref{cor:left_Kan_extension_barycentric_along_Yoneda}, the proof is short and straight forward. However, it refers to relatively basic results that, although known, do not seem readily available in the literature. Therefore, we present these basic results here.

We begin with the following two results, which say that a product of regular simplicial sets is regular and that a simplicial subset of a regular simplicial set is again regular. An argument is presented for the former of the two.
\begin{lemma}\label{lem:simplicial_subset_of_regular}
Let $X$ be a regular simplicial set and $A$ some simplicial subset. Then $A$ is regular.
\end{lemma}
\begin{proposition}\label{prop:product_of_regular}
Let
\[X=\prod _{j\in J}{X_j}\]
be a product of regular simplicial sets $X_j$, $j\in J$. Then $X$ is regular.
\end{proposition}
\begin{proof}[Proof of \cref{prop:product_of_regular}.]
Suppose $y\in X_n^\sharp$. For each $j\in J$, let $Y_j'$ be the image of the composite
\[\Delta [n-1]\xrightarrow{\delta _n} \Delta [n]\xrightarrow{\bar{y} } X\xrightarrow{pr_j} X_j.\]
Then we obtain the diagram
\begin{equation}
\label{eq:first_diagram_proof_prop_product_of_regular}
\begin{gathered}
\xymatrix{
\Delta [n-1] \ar[d]_{\delta _n} \ar[r] & Y_j' \ar[d] \ar@/^1.5pc/[ddr] \\
\Delta [n] \ar[r] \ar@/_1.5pc/[drr] & \Delta [n]\sqcup _{\Delta [n-1]}Y_j' \ar@{-->}[dr] \\
&& X_j
}
\end{gathered}
\end{equation}
in $sSet$, in which the canonical map from the pushout $\Delta [n]\sqcup _{\Delta [n-1]}Y_j'$ is degreewise injective as $X_j$ is regular.

The diagrams (\ref{eq:first_diagram_proof_prop_product_of_regular}) can be combined into the diagram
\begin{displaymath}
\xymatrix{
\Delta [n-1] \ar[d]_{\delta _n} \ar[r] & \prod _{j\in J}{Y_j'} \ar[d] \ar@/^1.5pc/[ddr] \\
\Delta [n] \ar[r] \ar@/_2pc/[drr]_{\bar{y} } & \prod _{j\in J}{(\Delta [n]\sqcup _{\Delta [n-1]}Y_j')} \ar[dr] \\
&& \prod _{j\in J}{X_j}
}
\end{displaymath}
that can be expanded to
\begin{equation}
\label{eq:second_diagram_proof_prop_product_of_regular}
\begin{gathered}
\xymatrix@C=0.9em@R=1.2em{
\Delta [n-1] \ar[d]_{\delta _n} \ar[r] & Y' \ar[d] \ar[r] & \prod _{j\in J}{Y_j'} \ar[d] \ar@/^2.5pc/[dddr]  \\
\Delta [n] \ar[r] \ar@/_2pc/[ddrrr]_{\bar{y} } \ar@/_1pc/[drr] & \Delta [n]\sqcup _{\Delta [n-1]}Y' \ar@{-->}[r] & \Delta [n]\sqcup _{\Delta [n-1]}(\prod _{j\in J}{Y_j')}) \ar@{-->}[d] \\
&& \prod _{j\in J}{(\Delta [n]\sqcup _{\Delta [n-1]}Y_j')} \ar[dr] \\
&&& \prod _{j\in J}{X_j}
}
\end{gathered}
\end{equation}
if we factor
\[\Delta [n-1]\to \prod _{j\in J}{Y_j'}\]
as a degreewise surjective map $\Delta [n-1]\to Y'$ followed by an inclusion.

Notice that $Y'$ is identified with the simplicial subset of $X$ that is generated by $y\delta _n$. It follows that $y$ is a regular simplex if the map
\[\Delta [n]\sqcup _{\Delta [n-1]}Y'\to X\]
is degreewise injective. This is true if the composite
\[\Delta [n]\sqcup _{\Delta [n-1]}Y'\to \Delta [n]\sqcup _{\Delta [n-1]}(\prod _{j\in J}{Y_j')})\to \prod _{j\in J}{(\Delta [n]\sqcup _{\Delta [n-1]}Y_j')}\]
is degreewise injective.

Assume that $w$ and $w'$ are different simplices of $\Delta [n]\sqcup _{\Delta [n-1]}Y'$ of the same degree, say of degree $q\geq 0$. We will prove that $w\mapsto e$ and $w'\mapsto e'$ are sent to different simplices $e$ and $e'$ in $\prod _{j\in J}{(\Delta [n]\sqcup _{\Delta [n-1]}Y_j')}$. There are three cases. The simplices $w$ and $w'$ can both be in the image of $Y'\to \Delta [n]\sqcup _{\Delta [n-1]}Y'$. It is also possible that neither of them are. By symmetry, the third possibility is that $w$ is in the image of $Y'\to \Delta [n]\sqcup _{\Delta [n-1]}Y'$ and that $w'$ is not.

Suppose $z\mapsto w$ and $z'\mapsto w'$ for some $q$-simplices $z$ and $z'$ of $Y'$. Then $Y'\to \prod _{j\in J}{Y_j'}$ maps $z\mapsto c$ and $z'\mapsto c'$ where $c$ and $c'$ are different as this map is an inclusion. Finally, the map
\[\prod _{j\in J}{Y_j'}\to \prod _{j\in J}{(\Delta [n]\sqcup _{\Delta [n-1]}Y_j')}\]
is degreewise injective as each simplicial set $Y_j'$, $j\in J$, is regular. Therefore, we get that $c\mapsto e$ and $c'\mapsto e'$ for different simplices $e$ and $e'$ in $\prod _{j\in J}{(\Delta [n]\sqcup _{\Delta [n-1]}Y_j')}$.

If neither $w$ nor $w'$ is in the image of $Y'\to \Delta [n]\sqcup _{\Delta [n-1]}Y'$, then we assume $b\mapsto w$ and $b'\mapsto w'$ for $q$-simplices $b$ and $b'$ of $\Delta [n]$. Choose some $j\in J$. The composite
\[\Delta [n]\to \prod _{j\in J}{(\Delta [n]\sqcup _{\Delta [n-1]}Y_j')}\xrightarrow{pr_j} \Delta [n]\sqcup _{\Delta [n-1]}Y_j'\]
sends $b$ and $b'$ to different simplices in $\Delta [n]\sqcup _{\Delta [n-1]}Y_j'$ as neither $b$ nor $b'$ is in the image of $N\delta _n$. Consequently, the first half of the composite maps $b\mapsto e$ and $b'\mapsto e'$ for different simplices $e$ and $e'$ in $\prod _{j\in J}{(\Delta [n]\sqcup _{\Delta [n-1]}Y_j')}$.

For the third case, assume that $z\mapsto w$ for some simplex $z$ in $Y'$ and that $w'$ is not in the image of $Y'\to \Delta [n]\sqcup _{\Delta [n-1]}Y'$. Then there is some simplex $b$ in $\Delta [n]$ such that $b'\mapsto w'$. Choose some $j\in J$. Consider the composites
\[\Delta [n-1]\to Y'\to \prod _{j\in J}{Y_j'}\xrightarrow{pr_j} Y_j'\]
and
\[\Delta [n]\to \prod _{j\in J}{(\Delta [n]\sqcup _{\Delta [n-1]}Y_j')}\xrightarrow{pr_j} \Delta [n]\sqcup _{\Delta [n-1]}Y_j'.\]
The first is the upper horizontal map in the cocartesian square in the $j$-th diagram (\ref{eq:first_diagram_proof_prop_product_of_regular}). The second is its cobase change along $N\delta _n$. As $b$ is not in the image of $N\delta _n$, it follows that the second of the two composites sends $b'$ to some simplex in $Y_j'$ that is not in the image of $Y_j'\to \Delta [n]\sqcup _{\Delta [n-1]}Y_j'$. Because the square
\begin{displaymath}
\xymatrix{
\prod _{j\in J}{Y_j'} \ar[d] \ar[r]^(.55){pr_j} & Y_j' \ar[d] \\
\prod _{j\in J}{(\Delta [n]\sqcup _{\Delta [n-1]}Y_j')} \ar[r]_(.58){pr_j} & \Delta [n]\sqcup _{\Delta [n-1]}Y_j'
}
\end{displaymath}
commutes, we see from (\ref{eq:second_diagram_proof_prop_product_of_regular}) that the image under $Y'\to \prod _{j\in J}{Y_j'}$ of $z$ is sent by $\prod _{j\in J}{Y_j'}\to \prod _{j\in J}{(\Delta [n]\sqcup _{\Delta [n-1]}Y_j')}$ to some $e$ that is different from $e'$ where $b'\mapsto e'$ under $\Delta [n]\to \prod _{j\in J}{(\Delta [n]\sqcup _{\Delta [n-1]}Y_j')}$.
\end{proof}
\noindent The results \cref{lem:simplicial_subset_of_regular} and \cref{prop:product_of_regular} yields the regularization functor, which is constructed thus.

Let $rsSet$ denote the full subcategory of $sSet$ whose objects are the regular simplicial sets. Given a simplicial set $X$, index a product over the quotient maps $X\to Y$ whose target $Y$ is regular. The product has as its factors the targets $Y$. We obtain a regular simplicial set $RX$ defined as the image of
\[X\to \prod _{f:X\to Y}{Y}\]
given by $x\mapsto (f(x))_f$. We say that $RX$ is the \textbf{regularization of $X$}. As the epimorphisms of simplicial sets are precisely the degreewise surjective maps and as every quotient map is degreewise surjective, the map $X\to RX$ is initial among the maps whose source is $X$ and whose target is regular.

The initial map becomes the unit of an adjunction in which $R$ is left adjoint to the inclusion $U:rsSet\to sSet$. One can in other words construct $R$ precisely as $D$ is constructed in \cite[Rem.~2.2.12]{WJR13}, except that non-singular simplicial sets is replaced with regular simplicial sets.

To prove \cref{cor:left_Kan_extension_barycentric_along_Yoneda}, we will also use the following basic result concerning Kan extensions. Note that we recycle the symbol $R$ for the purpose of stating and proving \cref{lem:Kan_extension_along_composite}.
\begin{lemma}\label{lem:Kan_extension_along_composite}
Consider a diagram
\[\mathscr{D} \xleftarrow{R} \mathscr{C} \xleftarrow{K} \mathscr{M} \xrightarrow{T} \mathscr{A}\]
where $\mathscr{M}$ is a small category and where $\mathscr{A}$ is cocomplete. Suppose the left Kan extension $Lan_{RK}T$ of $T$ along $RK$ exists.

If $R$ is fully faithful and admits a left adjoint functor $L:\mathscr{D} \to \mathscr{C}$, then the composite
\[Lan_KT=Lan_{RK}T\circ R\]
is the left Kan extension of $T$ along $K$.
\end{lemma}
\noindent Here, we follow the notation of \cite[§X]{ML98} closely as we will refer to results from that section in the proof.

Unfortunately, it seems that the context of \cref{lem:Kan_extension_along_composite} becomes clearest when we temporarily let $R$ denote the right adjoint indicated in the formulation of the lemma, rather than regularization. Then $R$ signifies \emph{right} and $L$ signifies \emph{left}. In this way, the case of \cref{lem:Kan_extension_along_composite} stands out from case of \cite[X.5~Thm.~1]{ML98}. However, the confusion should only be momentarily.

We are ready to prove the lemma.
\begin{proof}[Proof of \cref{lem:Kan_extension_along_composite}.]
Note that the left Kan extension $Lan_KT$ of $T$ along $K$ exists because $\mathscr{M}$ is small and because $\mathscr{A}$ is cocomplete \cite[§X.3~Cor.~2]{ML98}. By \cite[Ex.~X.4.3]{ML98}, the left Kan extension $Lan_R(Lan_KT)$ of $Lan_KT$ along $R$ exists as the left Kan extension $Lan_{RK}T$ exists. Moreover, we have that
\[Lan_R(Lan_KT)=Lan_{RK}T\]
by the same exercise.

We have natural transformations
\[\epsilon _K:T\Rightarrow (Lan_KT)\circ K\]
and
\[\epsilon _R:Lan_KT\Rightarrow Lan_R(Lan_KT)\circ R\]
that come with the two of our three Kan extensions. Next, let $\delta _R$ be the inverse of the map
\[(Lan_KT)\circ LR\xRightarrow{\cong } Lan_KT\]
that arises from the counit of the pair $(L,R)$. The counit $\epsilon _c:LRc\xrightarrow{\cong } c$ is an isomorphism as $R$ is fully faithful \cite[§IV.3~Thm.~1]{ML98}.

There is a (unique) natural transformation
\[\sigma _R:Lan_{RK}T\Rightarrow (Lan_KT)\circ L\]
such that the triangle on the left hand side in
\begin{equation}
\label{eq:first_diagram_proof_lem_Kan_extension_along_composite}
\begin{gathered}
\xymatrix@=1em{
& (Lan_{RK}T)\circ R \ar@/^/@{=>}[ddr]^(.55)\sigma \ar@{=>}[dd]^{\sigma _RR} \\
Lan_KT \ar@{=>}[ur]^(.4){\epsilon _R} \ar@{=>}[dr]_(.4){\delta _R} \\
& (Lan_KT)\circ LR \ar@{=>}[r]_(.63)\cong & Lan_KT
}
\end{gathered}
\end{equation}
commutes. The right hand side triangle in (\ref{eq:first_diagram_proof_lem_Kan_extension_along_composite}) was formed simply by letting $\sigma$ be the composite. Because $R$ is fully faithful, the natural transformation $\epsilon _R$ is a natural isomorphism \cite[§X.3~Cor.~3]{ML98}. This implies that $\sigma$ is a natural isomorphism and hence that $(Lan_{RK}T)\circ R$ is the left Kan extension of $T$ along $K$.
\end{proof}
\noindent With \cref{lem:Kan_extension_along_composite}, we have every result that we will use to establish our second corollary of \cref{thm:main_opt_triang}.

Similarly to the first corollary, we obtain the following.
\begin{corollary}\label{cor:left_Kan_extension_barycentric_along_Yoneda}
The composite
\[rsSet\xrightarrow{U} sSet\xrightarrow{B} nsSet\]
is a left Kan extension of $T$ along $\Upsilon$.
\end{corollary}
\begin{proof}
Let $(R,U)$ be the pair consisting of regularization and the inclusion. Because $Sd$ is the left Kan extension of $UT$ along $U\Upsilon$, the functor $SdU$ is the left Kan extension of $UT$ along $\Upsilon$ by \cref{lem:Kan_extension_along_composite}. The functor $DSdU$ is the left Kan extension of $T\cong DUT$ \cite[§IV.3~Thm.~1]{ML98} along $\Upsilon$ \cite[§X.5~Thm.~1]{ML98}. Now our result follows from \cref{thm:main_opt_triang}.
\end{proof}




