
\section{Degree zero}
\label{sec:dzero}


We make use of the following result. Let $Cat$ denote the category of small categories.
\begin{lemma}\label{lem:degree_zero_colimit_category_nerve}
Let $F:J\to Cat$ be a functor whose source is a small category. Let $\mathscr{L}$ be the colimit of $F$. If $X$ is the colimit of the composite diagram
\[J\xrightarrow{F} Cat\xrightarrow{N} sSet,\]
then the canonical map $X\to N\mathscr{L}$ is a bijection in degree $0$.
\end{lemma}
\begin{proof}
Let $O$ denote the functor $Cat\to Set$ that takes a small category to the set of its objects. Recall that $O$ has a right adjoint, namely the functor that takes a set $S$ to the indiscrete category $IS$. This is the category whose set of objects is precisely $S$ and that is such that each hom set is a singleton.

We also use the functor
\[sSet=Fun(\Delta ^{op},Set)\xrightarrow{(-)_0} Set\]
that sends a simplicial set to the set of its $0$-simplices. There is a natural bijection
\[O\mathscr{C} \xrightarrow{\cong} (N\mathscr{C} )_0,\]
that takes an element $c$ of the set $O\mathscr{C}$ of objects of a small category $\mathscr{C}$ to the simplex $[0]\to \mathscr{C}$ with $0\mapsto c$.

Because $O$ is cocontinous, we get a canonical function $O\mathscr{L} \to X_0$. As colimits in $sSet$ are formed degreewise it follows that this function is a bijection. There is also a canonical function $O\mathscr{L} \to (N\mathscr{L} )_0$, which by naturality must be the mentioned bijection. The induced map $X_0\to (N\mathscr{L} )_0$ fits into a triangle
\begin{displaymath}
\xymatrix{
O\mathscr{L} \ar[dr]_\cong \ar[rr]^\cong && (N\mathscr{L} )_0 \\
& X_0 \ar[ur] \\
}
\end{displaymath}
that commutes by the universal property of the colimit $O\mathscr{L}$. Hence, our claim that $X\to N\mathscr{L}$ is a bijection in degree $0$ is true.
\end{proof}
\noindent An application of the previous lemma is the following example.
\begin{example}\label{ex:pushout_poset_along_dwyer}
Let $F':J\to PoSet$ be a diagram
\begin{displaymath}
\xymatrix{
P \ar[d]_k \ar[r]^\varphi & R \\
Q
}
\end{displaymath}
where $k$ is a Dwyer map. As $PoSet$ is a reflective subcategory of $Cat$, it follows that $U:PoSet\to Cat$ preserves the pushout of $F'$ \cite[Lem.~5.6.4]{Th80}. If $Q\sqcup _PR$ is the colimit of $F=U\circ F'$, then \cref{lem:degree_zero_colimit_category_nerve} says that
the canonical map
\[NQ\sqcup _{NP}NR\to N(Q\sqcup _PR)\]
is a bijection in degree $0$.

In particular, if $k$ is the special Dwyer map
\[k=i_0:P\to P\times [1]=Q,\]
then the reduction map
\[cr:T(N\varphi )\to M(N\varphi )\]
is in general a bijection in degree $0$.
\end{example}

