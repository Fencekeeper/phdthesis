
\begin{abstract}
\noindent The Barratt nerve, denoted $B$, is the endofunctor that takes a simplicial set to the nerve of the poset of its non-degenerate simplices. The ordered simplicial complex $BSd\, X$, namely the Barratt nerve of the Kan subdivision $Sd\, X$, is a triangulation of the original simplicial set $X$ in the sense that there is a natural map $BSd\, X\to X$ whose geometric realization is homotopic to some homeomorphism. This is a refinement to the result that any simplicial set can be triangulated.

A simplicial set is said to be regular if each of its non-degenerate simplices is embedded along its $n$-th face. That $BSd\, X\to X$ is a triangulation of $X$ is a consequence of the fact that the Kan subdivision makes simplicial sets regular and that $BX$ is a triangulation of $X$ whenever $X$ is regular. In this paper, we argue that $B$, interpreted as a functor from regular to non-singular simplicial sets, is not just any triangulation, but in fact the best. We mean this in the sense that $B$ is the left Kan extension of barycentric subdivision along the Yoneda embedding.
\end{abstract}
