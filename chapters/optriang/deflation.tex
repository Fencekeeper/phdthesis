
\section{A deflation theorem}
\label{sec:deflation_thm}

In this section, we will prove a basic yet useful result concerning regular simplicial sets.

We begin with the following observation.
\begin{lemma}\label{lem:property_of_regular_simplicial_set}
Let $y$ be a regular non-degenerate simplex, say of degree $n$, of some simplicial set. Assume that $y\mu$ and $y\nu$ are faces of $y$ such that the last vertex of $y$ is a vertex of one of them. If
\[(y\mu )^\sharp = (y\nu )^\sharp ,\]
then $\mu =\nu$.
\end{lemma}
\begin{proof}
Let $Y$ denote the simplicial subset that is generated by $y$ and let $Y'$ be generated by $y\delta _{n}$. Then the canonical map
\[\Delta [n]\sqcup _{\Delta [n-1]}Y'\xrightarrow{\cong } Y\]
is an isomorphism as $y$ is regular. We want to think of the simplices $y\mu$ and $y\nu$ of $Y$ as simplices of $\Delta [n]\sqcup _{\Delta [n-1]}Y'$.

Note that the isomorphism above implies that $y\varepsilon _n\neq y\varepsilon _j$ for all $j$ with $0\leq j<n$. By the assumption that the last vertex of $y$ is a vertex of $y\mu$ or of $y\nu$ we have that $n$ is in the image of at least one of the face operators $\mu$ and $\nu$. Say that $n$ is in the image of $\mu$. Then $y\mu =(y\mu )^\sharp$, and $y\mu$ is not in the image of
\[Y'\to \Delta [n]\sqcup _{\Delta [n-1]}Y'.\]
From $(y\mu )^\sharp =(y\nu )^\sharp$ it follows that $(y\nu )^\sharp$ is not in the image of this map, hence $y\nu$ is not. As $y\nu$ is the image of $\nu$ under
\[\Delta [n]\to \Delta [n]\sqcup _{\Delta [n-1]}Y'\]
it follows that $\nu$ is not in the image of $N\delta _n$, hence $n$ is in the image of $\nu$. This means that $y\nu =(y\nu )^\sharp$. Now it follows that $y\mu =y\nu$, so $\mu$ and $\nu$ must have the same source, say $[k]$. The function
\[\Delta [n]_k\to (\Delta [n]\sqcup _{\Delta [n-1]}Y')_k\]
is injective on the complement of the image of $(N\delta _n)_k$, which implies
\[\mu =\nu .\]
\end{proof}
\noindent Now, \cref{lem:property_of_regular_simplicial_set} may be intuitively obvious. However, the next result may not be obvious.

Consider a $2$-simplex of some regular simplicial set such that the non-degenerate parts of the first face and the second face are equal. Then the $2$-simplex is degenerate. Moreover, its non-degenerate part is equal to the two previously mentioned non-degenerate parts. In this sense, the $2$-simplex is deflated. One can say the following, in general.
\begin{proposition}\label{prop:deflation_theorem}
Let $X$ be a regular simplicial set and $y$ a simplex, say of degree $n$. Suppose $[n]$ the union of the images of two face operators $\mu$ and $\nu$ and that neither image is contained in the other. If
\[(y\mu )^\sharp =(y\nu )^\sharp ,\]
then $y$ is degenerate with non-degenerate part equal to the non-degenerate parts of $y\mu$ and $y\nu$.
\end{proposition}
\begin{proof}
Note that \cref{lem:property_of_regular_simplicial_set} immediately implies that $y$ is degenerate. Now, define
\[\alpha =y^\flat \mu\]
and take the unique factorization of
\[\alpha =\alpha ^\sharp \alpha ^\flat\]
into a degeneracy operator $\alpha ^\flat$ followed by a face operator $\alpha ^\sharp$. Similarly, we write
\[y^\flat \nu =\beta =\beta ^\sharp \beta ^\flat .\]
Now, the union of the images of the face operators $\alpha ^\sharp$ and $\beta ^\sharp$ is equal to their common target as the pair $(\mu ,\nu )$ has this property.

The left hand side of the equation $(y\mu )^\sharp =(y\nu )^\sharp$ can be written
\[(y^\sharp y^\flat \mu )^\sharp =(y^\sharp \alpha ^\sharp \alpha ^\flat )^\sharp =(y^\sharp \alpha ^\sharp )^\sharp\]
and the right hand side can be written
\[(y^\sharp y^\flat \nu )^\sharp =(y^\sharp \beta ^\sharp \beta ^\flat )^\sharp =(y^\sharp \beta ^\sharp )^\sharp.\]
By \cref{lem:property_of_regular_simplicial_set}, it follows that $\alpha ^\sharp =\beta ^\sharp$. As the union of the images of $\alpha ^\sharp$ and $\beta ^\sharp$ is equal to their common target it follows that both of the face operators are equal to the identity. This means that
\[(y^\sharp \alpha ^\sharp )^\sharp =(y^\sharp )^\sharp =y^\sharp\]
and the leftmost expression is equal to $(y\mu )^\sharp$. This concludes the proof.
\end{proof}




