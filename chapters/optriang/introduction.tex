\section{Introduction}
\label{sec:intro}

Not every CW complex can be triangulated \cite{Me67}, but simplicial sets can. The latter fact is largely due to Barratt \cite{Ba56}, but a correct proof was first given by Fritsch and Puppe in \cite{FP67}. One can prove it by arguing that all regular CW complexes are trianguable, that regular simplicial sets give rise to regular CW complexes and that the geometric realization of the last vertex map $d_X:Sd\, X\to X$ \cite[§7]{Ka57}, from the Kan subdivision $Sd\, X$ of $X$ \cite[§7]{Ka57}, is homotopic to a homeomorphism. Fritsch and Piccinini \cite[pp.~208--209]{FP90} tell the whole story in detail.

By a regular simplicial set, we mean the following.
\begin{definition}
Let $X$ be a simplicial set and suppose $y$ a non-degenerate simplex, say of dimension $n$. The simplicial subset of $X$ generated by $y\delta _n$ is denoted $Y'$. We can then consider the diagram
\begin{displaymath}
\xymatrix@=1em{
\Delta [n-1] \ar[d]_{\delta _n} \ar[r] & Y' \ar[d] \ar@/^1.5pc/[ddr] \\
\Delta [n] \ar@/_1pc/[drr] \ar[r] & \Delta [n]\sqcup _{\Delta [n-1]}Y' \ar[dr] \\
&& X
}
\end{displaymath}
in $sSet$ in which the upper left hand square is cocartesian. We say that $y$ is \textbf{regular} \cite[p.~208]{FP90} if the canonical map from the pushout is degreewise injective.
\end{definition}
\noindent We say that a simplicial set is \textbf{regular} if its non-degenerate simplices are regular.

There is a refinement to the result that all simplicial sets can be triangulated, as explained by Fritsch and Piccinini \cite[Ex.~5--8,~pp.~219--220]{FP90}. The triangulation of a given regular CW-complex described in Theorem 3.4.1 in \cite{FP90}, which is the barycentric subdivision when the CW-complex is the geometric realization of a simplicial complex, can be adapted to the setting of simplicial sets. The adaptation is an endofunctor $B:sSet\to sSet$ of simplicial sets, which is in \cite[p.~35]{WJR13} referred to as the Barratt nerve.

Let $N:Cat\to sSet$ be the fully faithful nerve functor from small categories to simplicial sets. Let $X^\sharp$ be the partially ordered set (poset) of non-degenerate simplices of $X$ with $y\leq x$ when $y$ is a face of $x$. In general, a poset $(P,\leq )$ can be thought of as a small category in the following way. Let the objects be the elements of $P$ and let there be a morphism $p\to p'$ whenever $p\leq p'$. The full subcategory of $Cat$ whose objects are the ones that arise from posets, we denote $PoSet$. The poset $X^\sharp$ is in some sense the smallest simplex category of $X$. The simplicial set $BX=NX^\sharp$ is the \textbf{Barratt nerve} of $X$.

There is a canonical map
\[b_X:Sd\, X\to BX\]
as explained in \cite[p.~37]{WJR13}. It is natural and expresses the viewpoint that $Sd$ is the left Kan extension of barycentric subdivision of standard simplices along the Yoneda embedding \cite[X.3~(10)]{ML98}. By this viewpoint, even the Kan subdivision performs barycentric subdivision on standard simplices \cite[X.3~Cor.~3]{ML98} as the Yoneda embedding is in particular fully faithful. Moreover, the map $b_X$ is degreewise surjective in general \cite[Lem.~2.2.10, p.~38]{WJR13} and an isomorphism if and only if $X$ is non-singular \cite[Lem.~2.2.11, p.~38]{WJR13}.

The Yoneda lemma puts the $n$-simplices $x$, $n\geq 0$, of a simplicial set $X$ in a natural bijective correspondence $x\mapsto \bar{x}$ with the simplicial maps $\bar{x} :\Delta [n]\to X$. Here, $\Delta [n]$ denotes the standard $n$-simplex. We refer to $\bar{x}$ as the \textbf{representing map} of the simplex $x$.
\begin{definition}
A simplicial set is \textbf{non-singular} if the representing map of each of its non-degenerate simplex is degreewise injective. Otherwise it is said to be \textbf{singular}.
\end{definition}
\noindent The inclusion $U$ of the full subcategory $nsSet$ of non-singular simplicial sets admits a left adjoint $D:sSet\to nsSet$, which is called desingularization \cite[Rem.~2.2.12]{WJR13}.

The map $b_X$ factors through the unit $\eta _{Sd\, X}:Sd\, X\to UD(Sd\, X)$ of the adjunction $(D,U)$. This gives rise to a degreewise surjective map
\[t_X:DSd\, X\to BX\]
that is a bijection in degree $0$. As $\eta _{Sd\, X}$ is degreewise surjective, we obtain a natural transformation $t$ between functors $sSet\to nsSet$. Our main result is the following.
\begin{theorem}\label{thm:main_opt_triang}
The natural map $t_X:DSd\, X\to BX$ is an isomorphism whenever $X$ is regular.
\end{theorem}
\noindent We will begin the proof of our main result in \cref{sec:mapcyl}.

A notion referred to as the reduced mapping cylinder \cite[§2.4]{WJR13} appears in the proof of \cref{thm:main_opt_triang}. Let $\varphi :P\to R$ be an order-preserving function between posets. The nerve
\[M(N\varphi )=N(P\times [1]\sqcup _PR)\]
of the pushout in the category of posets of the diagram
\begin{equation}
\begin{gathered}
\xymatrix{
P \ar[d]_{i_0} \ar[r]^\varphi & R \\
P\times [1]
}
\end{gathered}
\end{equation}
is known as the \textbf{(backwards) reduced mapping cylinder} of $N\varphi$ \cite[Def.~2.4.4]{WJR13}. If we think of posets as small categories as above and use the nerve to yield a diagram in $sSet$, then we obtain the pushout $T(N\varphi )$ known as the \textbf{(backwards) topological mapping cylinder} together with a \textbf{cylinder reduction map} \cite[Def.~2.4.5]{WJR13}
\[cr:T(N\varphi )\to M(N\varphi ).\]
In \cite[§2.4]{WJR13} the reduced mapping cylinder is introduced in full generality, meaning for an arbitrary simplicial map and not just for the nerve of an order-preserving function between posets. We refer to that source for the general construction.

The cylinder reduction map gives rise to a canonical map
\[dcr:DT(N\varphi )\to M(N\varphi )\]
from the desingularized toplogical mapping cylinder. \cref{thm:main_opt_triang} relies upon the following result, as we explain in \cref{sec:mapcyl}.
\begin{theorem}\label{thm:barratt_nerve_rep_map_dcr_iso}
Let $X$ be a regular simplicial set. For each $n\geq 0$ and each $n$-simplex $y$, the canonical map
\[dcr:DT(B(\bar{y} )\xrightarrow{\cong } M(B(\bar{y} ))\]
is an isomorphism.
\end{theorem}
\noindent This result does not seem to follow easily from the theory of \cite[§§2.4--2.5]{WJR13}, although it can essentially be deduced from \cite[Cor.~2.5.7]{WJR13} that $dcr$ is degreewise surjective and although $dcr$ is easily seen to be a bijection in degree $0$.

\cref{thm:barratt_nerve_rep_map_dcr_iso} is a refinement to one of the statements of Lemma $2.5.6$ of \cite[p.~71]{WJR13}. In \cref{sec:comparison}, we discuss a result related to \cref{thm:barratt_nerve_rep_map_dcr_iso}, but whose proof is easier. Namely, \cref{prop:cones_vs_mapping_cylinders} says that the desingularization of the cone on $NP$ is the reduced mapping cylinder of the unique map $NP\to \Delta [0]$, for every poset $P$.

The intuition behind \cref{thm:main_opt_triang} is as follows. One can look at $X=Sd\, Y$ for $Y$ some slightly singular example such as when $Y$ is the result of collapsing some $(n-1)$-dimensional face of a standard $n$-simplex. Another example is the model $Y=\Delta [n]/\partial \Delta [n]$ of the $n$-sphere for $0\leq n\leq 2$. When $n=0$ and $n=1$, it is clear that $t_X$ is an isomorphism. However, an argument is required for the case when $n=2$. These computations are done in \cref{sec:examples}. Simple, but representative examples point in the same direction, namely that $t_X$ seems to be an isomorphism whenever $X$ is the Kan subdivision of some simplicial set $Y$.

If one is tempted to ask whether $t_X$ is an isomorphism whenever $X$ is a Kan subdivision, then it is no great leap to ask whether $t_X$ is an isomorphism for every regular simplicial set $X$. The book ``Spaces of PL manifolds and categories of simple maps'' \cite[Rem.~2.2.12,~p.~40]{WJR13} asks precisely this question. Our main result is thus an affirmative answer. There is a close relationship between regular simplicial sets and the simplicial sets that are Kan subdivisions. In fact, the Kan subdivision of every simplicial set is regular \cite[Prop.~4.6.10, p.~208]{FP90}.

In \cref{sec:conseq}, we discuss consequences of our main result. We explain how \cref{thm:main_opt_triang} follows from \cref{thm:barratt_nerve_rep_map_dcr_iso}, in \cref{sec:mapcyl}. It seems fitting that we refer forward to the various parts of the proof of \cref{thm:barratt_nerve_rep_map_dcr_iso} from \cref{sec:mapcyl} instead of from this introduction, so this is what we will do. Each section of this paper that follows \cref{sec:conseq} is essentially part of the proof of \cref{thm:main_opt_triang} and of \cref{thm:barratt_nerve_rep_map_dcr_iso}, except \cref{sec:cones}. The latter presents \cref{prop:cones_vs_mapping_cylinders}, which is a result on cones. It can be viewed as related to \cref{thm:barratt_nerve_rep_map_dcr_iso}.



