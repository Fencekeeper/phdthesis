

\section{Rigidity of the prism}
\label{sec:prism}

We give a proof that $X^{\Delta [1]}$ is non-singular whenever $X$ is non-singular. This is the claim presented in \cref{prop:standard_1_simplex_as_exponent}. An informal way of stating this result is to say that prisms on non-singular simplicial sets are very rigid. Recall that \cref{sec:arbexp} explains how to derive \cref{thm:arbitrary_exponent} from \cref{prop:standard_1_simplex_as_exponent}. Thus the work of this section finishes the proof of our main result.

For convenience, we introduce some terminology and notation before we present the proof. An injective operator is said to be a \textbf{face operator} and a surjective operator is said to be a \textbf{degeneracy operator}. Special face operators are the \textbf{elementary face operators} $\delta ^n_i:[n-1]\to [n]$ that omit the index $i$ and \textbf{vertex operators} $\varepsilon ^n_i:[0]\to [n]$ that hit the indices $i$. Special degeneracy operators are the \textbf{elementary degeneracy operators} $\sigma ^n_i:[n+1]\to [n]$ that send $i$ and its successor $i+1$ to $i$. Frequently, we omit the upper index in the notation. Similar to the terminology in \cite{WJR13}, we will refer to $\delta ^n_n\dots \delta ^q_q:[q-1]\to [n]$, $0<q\leq n$, as the \textbf{$q$-th front face} of $[n]$ and to $\delta ^n_p\dots \delta ^{n-p}_0:[n-(p+1)]\to [n]$, $0\leq p<n$, as the \textbf{$p$-th back face} of $[n]$.

A face operator or degeneracy operator is \textbf{proper} if it is not the identity. Consider a simplicial set. A simplex $y$ is a \textbf{(proper) face} of another simplex $x$ if $y=x\mu$ for a (proper) face operator $\mu$. Analogously, a simplex $y$ is a \textbf{(proper) degeneracy} of another simplex $x$ if $y=x\rho$ for a (proper) degeneracy operator $\rho$. A simplex is \textbf{degenerate} if it is a proper degeneracy of some simplex. Otherwise, it is said to be \textbf{non-degenerate}.

In the proof, we will use the Eilenberg-Zilber lemma \cite[Thm.~4.2.3]{FP90}, which says that any simplex $x$ of any simplicial set $X$ is uniquely a degeneration $x=x^\sharp x^\flat$ of some non-degenerate simplex $x^\sharp$. We say that $x^\sharp$ is the \textbf{non-degenerate part} of $x$, following \cite{WJR13}, and that $x^\flat$ is the \textbf{degenerate part} of $x$. Note that $x$ and $x^\sharp$ are objects in the category $\Delta \downarrow X$ while $x^\flat$ can be regarded as a morphism $x\to x^\sharp$. Thus the terminology is not perfect, however it is useful. According to the Yoneda lemma, the $n$-simplices $x$ of a simplicial set $X$ are in natural bijective correspondence $x\mapsto \bar{x}$ with the simplicial maps $\Delta [n]\to X$. The map $\bar{x}$ is the \textbf{representing map} of $x$. We say that a simplex is \textbf{embedded} if its representing map is degreewise injective.

Because of the new terminology, we get a shorter definition of \emph{non-singular} in the second condition of \cref{lem:non-singular_equivalent_criteria}, below. Furthermore, there is another formulation that is useful in the proof of \cref{prop:standard_1_simplex_as_exponent}, though a bit awkward. It is given as the third condition \cref{lem:non-singular_equivalent_criteria}
\begin{lemma}\label{lem:non-singular_equivalent_criteria}
The following statements are equivalent.
\begin{enumerate}
\item{The simplicial set $X$ is non-singular.}
\item{Each non-degenerate simplex of $X$ is embedded.}
\item{Eeach simplex of $X$ is degenerate provided its vertices are not pairwise distinct.}
\end{enumerate}
\end{lemma}
\noindent The equivalence of the second and third statement is somewhat refined by the next lemma.
\begin{lemma}\label{lem:degenerate_part_factorization_through}
Let $X$ be a non-singular simplicial set and $x$ some simplex with $z\varepsilon _k=z\varepsilon _l$. Then the degenerate part $x^\flat$ of $x$ factors uniquely through the degeneracy operator $\sigma _k\dots \sigma _{l-1}$.
\end{lemma}
\begin{proof}
Write $\rho =\sigma _k\dots \sigma _{l-1}$. The uniqueness of a factorization of $x^\flat$ through $\rho$ is automatic as $\rho$ is epic in $Cat$. It is the existence part that requires an argument.

Because $X$ is non-singular it follows that the non-degenerate part $x^\sharp$ is embedded, which is the same as saying that its vertices are pairwise distinct. This means that $x^\flat (k)=x^\flat (l)$. As $x^\flat$ is order-preserving, it follows that $x^\flat (j)=x^\flat (k)$ if $k\leq j\leq l$. Thus $\rho (i)=\rho (j)$ implies $x^\flat (i)=x^\flat (j)$. Take a section $\mu$ of $\rho$. We get that $x^\flat =(x^\flat \mu )\rho$.
\end{proof}
\noindent \cref{lem:degenerate_part_factorization_through} will be used to break down the proof of \cref{prop:standard_1_simplex_as_exponent} into two parts.

If $x$ is some simplex, say of degree $n$, whose degenerate part factors through the elementary degeneracy operator $\sigma _k$ for some $k$ with $0\leq k<n$, then we will say that $x$ \textbf{splits off} $\sigma _k$. In particular, if $X$ is non-singular and if $x$ is a simplex of $X$ such that $x\varepsilon _k=x\varepsilon _{k+1}$, then $x$ splits off $\sigma _k$ according to \cref{lem:degenerate_part_factorization_through}.

The canonical identification
\[N([n]\times [1])\xrightarrow{\cong } \Delta [n]\times \Delta [1]\]
gives us a preferred set of generators of the prism $\Delta [n]\times \Delta [1]$, namely the $n+1$ non-degenerate
$(n+1)$-simplices
\[\gamma ^{n+1}_j:[n+1]\rightarrow [n]\times [1],\]
$0\leq j\leq n$,
given by
\begin{displaymath}
\gamma ^{n+1}_j(i)=
\begin{cases}
(i,0), & 0 \leq i \leq j \\ 
(i-1,1), & j < i \leq n.
\end{cases}
\end{displaymath}
Coming from the diagram
\begin{displaymath}
\xymatrix{
& \dots \ar[r] & (j,1) \ar[r] & (j+1,1) \ar[r] & \dots \ar[r] & (n,1) \\
(0,0) \ar[r] & \dots \ar[r] & (j,0) \ar[u] \ar[r] \ar[ur] & (j+1,0) \ar[r] \ar[u] & \dots
}
\end{displaymath}
are the conditions
\begin{equation}\label{Equation_generator_compatibility}
\gamma ^{n+1}_j\delta _{j+1}=\gamma ^{n+1}_{j+1}\delta _{j+1}
\end{equation}
for $0\leq j\leq n$. These conditions, which can be thought of glueing conditions for constructing the prism from $n+1$
copies of the standard $(n+1)$-simplex, generate all relations that the generators satisfy.

We are done with the setup and are ready to prove \cref{prop:standard_1_simplex_as_exponent}. Suppose $X$ non-singular. Keep in mind the third and equivalent way to state this, as formulated in \cref{lem:non-singular_equivalent_criteria}. The proof is divided into two parts, the first of which is the following result.
\begin{lemma}\label{lem:part1_standard_1_simplex_as_exponent}
Assume that $\Phi$ is an $n$-simplex of $X^{\Delta [1]}$ such that the $k$-th vertex and the $l$-th vertex are equal, for some $k$ and some $l$ with $0\leq k<l\leq n$. Then
\[\Phi \varepsilon _k=\Phi \varepsilon _{k+1}=\dots =\Phi \varepsilon _l.\]
\end{lemma}
\noindent The second part is \cref{lem:part2_standard_1_simplex_as_exponent}, where we prove that any given $n$-simplex $\Phi$ of $X^{\Delta [1]}$ is degenerate if it is such that the $k$-th vertex is equal to the $(k+1)$-th vertex, for some $k$ with $0\leq k<n$.

Thus, by \cref{lem:part1_standard_1_simplex_as_exponent} and \cref{lem:part2_standard_1_simplex_as_exponent}, any simplex of $X^{\Delta [1]}$ is degenerate provided its vertices are not pairwise distinct. Lemma \cref{lem:non-singular_equivalent_criteria} then says that $X^{\Delta [1]}$ is non-singular. We can therefore conclude that \cref{prop:standard_1_simplex_as_exponent} holds when we have proven the two lemmas.

\begin{proof}[Proof of \cref{lem:part1_standard_1_simplex_as_exponent}.]
Suppose $\Phi$ an $n$-simplex of $X^{\Delta [1]}$ such that $\Phi\varepsilon _k=\Phi \varepsilon _l$ for some $k$ and some $l$ with $0\leq k<l\leq n$. What is immediately noticeable is that the composite of $\Phi$ with the inclusion of the bottom of the prism is an $n$-simplex
\[x_0=\Phi \circ (id,N\varepsilon _0)\]
of $X$ whose $k$-th and $l$-th vertex are also equal. Doing something similar at the top of the prism we get a simplex
$x_1=\Phi \circ (id,N\varepsilon _1)$ with $x_1\varepsilon _k=x_1\varepsilon _l$.

From \cref{lem:degenerate_part_factorization_through} it follows that the degenerate part $x_0^\flat$ of $x_0$ factors uniquely through $\sigma _k\dots \sigma _{l-1}$. Thus we can write
\begin{displaymath}
\begin{array}{rcl}
x_0 & = & y_0\sigma _k\dots \sigma _{l-1} \\
x_1 & = & y_1\sigma _k\dots \sigma _{l-1}
\end{array}
\end{displaymath}
for some $(k+n-l)$-simplices $y_0$ and $y_1$ of $X$.

Suppose $k\leq j<l$. Writing $x_0$ and $x_1$ as degenerations indicates that the $(n+1)$-simplices $\Phi (\gamma ^{n+1}_{j+1})$ and $\Phi (\gamma ^{n+1}_j)$ of $X$ must be degenerate. To answer how they are degenerate, form the left hand cartesian square in the following diagram.
\begin{displaymath}
\xymatrix{
\Delta [n+1] \ar[r]^(.45){\gamma ^{n+1}_{j+1}} & \Delta [n]\times \Delta [1] \ar[r]^(.6)\Phi & X \\
\Delta [j+1] \ar[u] \ar[r] & \Delta [n] \ar[u]^{(id,N\varepsilon _0)} \ar[ur]_{x_0} \ar[r]_(.4){N(\sigma _k\dots \sigma _{l-1})} & \Delta [k+n-l] \ar[u]_{y_0}
}
\end{displaymath}
The canonical map $\Delta [j+1]\to \Delta [n+1]$ is then induced by the $(j+2)$-th front face of $[n+1]$ and the canonical map $\Delta [j+1]\to \Delta [n]$ is induced by the $(j+2)$-th front face of $[n]$.

The above implies that the $j$-th and the $(j+1)$-th vertex of $\Phi (\gamma ^{n+1}_{j+1})$ are equal. A similarly constructed diagram involving $x_1$, $y_1$, $(id,N\varepsilon _1)$ and $\Phi (\gamma ^{n+1}_j)$ shows that the $(j+1)$-th and the $(j+2)$-th vertex of $\Phi (\gamma ^n_j)$ are equal.

As a consequence of the previous paragraph, we will argue that the $j$-th and the $(j+1)$-th vertex of the $n$-simplex $\Phi$ of $X^{\Delta [1]}$ are equal. They are the vertices of the $1$-simplex
\[\Delta [1]\times \Delta [1]\xrightarrow{N\mu \times id} \Delta [n]\times \Delta [1]\xrightarrow{\Phi } X,\]
of $X^{\Delta [1]}$ where $\mu$ is given by $0\mapsto j$ and $1\mapsto j+1$.

We can view the vertices $\Phi \varepsilon _j$ and $\Phi \varepsilon _{j+1}$ of the simplex $\Phi$ of $X^{\Delta [1]}$ as
$1$-simplices of $X$. When we do, they fit into the commutative diagram
\begin{displaymath}
 \xymatrix{
 & \Delta [2] \ar[dr]^{\gamma ^2_1} && \Delta [1] \ar[ll]_{\delta _0} \ar[dr]^{\Phi \varepsilon _{j+1}} \\
 \Delta [1] \ar[ur]^{\delta _1} \ar[dr]_{\delta _1} && \Delta [1]\times \Delta [1] \ar[rr]^(.6){\Phi \circ (N\mu \times id)} && X \\
 & \Delta [2] \ar[ur]_{\gamma ^2_0} && \Delta [1] \ar[ll]^{\delta _2} \ar[ur]_{\Phi \varepsilon _j}
 }
\end{displaymath}
that establishes $\Phi \varepsilon _{j+1}$ as a face of the $2$-simplex
\[z_1=\Phi \circ (N\mu \times 1)\circ \gamma ^2_1\]
and $\Phi \varepsilon _j$ as a face of the $2$-simplex
\[z_0=\Phi \circ (N\mu \times 1)\circ \gamma ^2_0,\]
in such a way that $z_1\delta _1=z_0\delta _1$.

Recall that the $j$-th and the $(j+1)$-th vertex of the simplex $\Phi (\gamma ^n_{j+1})$ of $X$ are equal. This implies that
\[z_1=w_1\sigma _1.\]
Similarly, the $(j+1)$-st and the $(j+2)$-nd vertex of $\Phi (\gamma ^n_j)$ are equal, implying that $z_0=w_0\sigma _0$. It follows that $\Phi \varepsilon _j=\Phi \varepsilon _{j+1}$ as $\delta _1$ and $\delta _0$ are sections of $\sigma _0$ and $\delta _1$ and $\delta _2$ are sections of $\sigma _1$.
\end{proof}

\begin{lemma}\label{lem:part2_standard_1_simplex_as_exponent}
Let $\Phi$ be an $n$-simplex of $X^{\Delta [1]}$ such that the $k$-th vertex is equal to the $(k+1)$-th vertex, for some $k$ with $0\leq k<n$. Then there is an $(n-1)$-simplex $\Psi$ such that $\Phi =\Psi \sigma _k$.
\end{lemma}
\begin{proof}
For the purpose of constructing $\Psi$ we apply $N\sigma _k \times id$ to the elements of the preferred set $\{ \gamma ^{n+1}_0$, $\dots$, $\gamma ^{n+1}_n\}$ of generators of the prism. The result of the calculation is the set of equations
\begin{displaymath}
(N\sigma _k\times id)(\gamma ^{n+1}_j)=
\begin{cases}
\gamma ^n_j\sigma _{k+1}, & 0 \leq j \leq k \\ 
\gamma ^n_{j-1}\sigma _k, & k < j \leq n.
\end{cases}
\end{displaymath}
Should $\Psi$ exist, then it must therefore satisfy
\begin{displaymath}
\Phi (\gamma ^{n+1}_j)=
\begin{cases}
\Psi (\gamma ^n_j)\sigma _{k+1}, & 0 \leq j \leq k \\ 
\Psi (\gamma ^n_{j-1})\sigma _k, & k < j \leq n.
\end{cases}
\end{displaymath}
As $\delta _{k+1}$ is a section of both $\sigma _k$ and $\sigma _{k+1}$ we are lead to define a function
\[\psi :\{ \gamma ^n_0,\dots ,\gamma ^n_{n-1}\} \to X_n\]
by
\begin{displaymath}
\psi (\gamma ^n_j)=
\begin{cases}
\Phi (\gamma ^{n+1}_j)\delta _{k+1}, & 0\leq j\leq k \\ 
\Phi (\gamma ^{n+1}_{j+1})\delta _{k+1}, & k<j<n
\end{cases}
\end{displaymath}
that specifies where $\Psi$ sends the generators, if it exists.

Note the following regarding the definition of $\psi$. First, we have made the choices of the section $\delta _{k+1}$ of $\sigma _{k+1}$ and the section $\delta _{k+1}$ of $\sigma _k$. These choices seem to make the argument below as simple as possible. Second, we have that
\[\psi (\gamma ^n_k)=\Phi (\gamma ^{n+1}_k)\delta _{k+1}=\Phi (\gamma ^{n+1}_k\delta _{k+1})=\Phi (\gamma ^{n+1}_{k+1}\delta _{k+1})=\Phi (\gamma ^{n+1}_{k+1})\delta _{k+1}\]
due to (\ref{Equation_generator_compatibility}). This ensures that there is some compatibility between the two clauses of the definition of $\psi$ by cases. We take advantage of the equation below.

Crucially, the function $\psi$ obeys the compatibility criterion
\begin{equation}\label{equation_compatibility_def_psi}
 \psi (\gamma ^n_j)\delta _{j+1}=\psi (\gamma ^n_{j+1})\delta _{j+1}
\end{equation}
for $0\leq j<n-1$, as we now explain. There are three cases. Either $j<k$, $j=k$ or $j>k$.

First, we verify (\ref{equation_compatibility_def_psi}) in the case when $j=k$. For this we use (\ref{Equation_generator_compatibility}) and the general rule $\delta _i\delta _j=\delta _j\delta _{i-1}$ for $j<i$. We get that
\begin{displaymath}
\begin{array}{rcl}
\psi (\gamma ^n_k)\delta _{k+1} & = & (\Phi (\gamma ^{n+1}_k)\delta _{k+1})\delta _{k+1} \\
& = & (\Phi (\gamma ^{n+1}_{k+1})\delta _{k+1})\delta _{k+1} \\
& = & \Phi (\gamma ^{n+1}_{k+1})(\delta _{k+1}\delta _{k+1}) \\
& = & \Phi (\gamma ^{n+1}_{k+1})(\delta _{k+2}\delta _{k+1}) \\
& = & (\Phi (\gamma ^{n+1}_{k+1})\delta _{k+2})\delta _{k+1} \\
& = & (\Phi (\gamma ^{n+1}_{k+1}\delta _{k+2}))\delta _{k+1} \\
& = & (\Phi (\gamma ^{n+1}_{k+2}\delta _{k+2}))\delta _{k+1}
\end{array}
\end{displaymath}
and that
\begin{displaymath}
\begin{array}{rcl}
\psi (\gamma ^n_{k+1})\delta _{k+1} & = & (\Phi (\gamma ^{n+1}_{k+2})\delta _{k+1})\delta _{k+1} \\
& = & \Phi (\gamma ^{n+1}_{k+2})(\delta _{k+1}\delta _{k+1}) \\
& = & \Phi (\gamma ^{n+1}_{k+2})(\delta _{k+2}\delta _{k+1}),
\end{array}
\end{displaymath}
which confirms that (\ref{equation_compatibility_def_psi}) holds in the case when $j=k$.

Second, consider the case when $j<k$. We get that
\begin{displaymath}
\begin{array}{rcl}
\psi (\gamma ^n_j)\delta _{j+1} & = & (\Phi (\gamma ^{n+1}_j)\delta _{k+1})\delta _{j+1} \\
& = & \Phi (\gamma ^{n+1}_j)(\delta _{k+1}\delta _{j+1}) \\
& = & \Phi (\gamma ^{n+1}_j)(\delta _{j+1}\delta _k) \\
& = & (\Phi (\gamma ^{n+1}_j)\delta _{j+1})\delta _k \\
& = & (\Phi (\gamma ^{n+1}_j\delta _{j+1}))\delta _k \\
& = & (\Phi (\gamma ^{n+1}_{j+1}\delta _{j+1}))\delta _k
\end{array}
\end{displaymath}
and that
\begin{displaymath}
\begin{array}{rcl}
\psi (\gamma ^n_{j+1})\delta _{j+1} & = & (\Phi (\gamma ^{n+1}_{j+1})\delta _{k+1})\delta _{j+1} \\
& = & \Phi (\gamma ^{n+1}_{j+1})(\delta _{k+1} \delta _{j+1}) \\
& = & \Phi (\gamma ^{n+1}_{j+1})(\delta _{j+1}\delta _k), \\
\end{array}
\end{displaymath}
which confirms that (\ref{equation_compatibility_def_psi}) holds in the case when $j<k$.

Third, consider the case when $j>k$. We get that
\begin{displaymath}
\begin{array}{rcl}
\psi (\gamma ^n_j)\delta _{j+1} & = & (\Phi (\gamma ^{n+1}_{j+1})\delta _{k+1})\delta _{j+1} \\
& = & \Phi (\gamma ^{n+1}_{j+1})(\delta _{k+1}\delta _{j+1}) \\
& = & \Phi (\gamma ^{n+1}_{j+1})(\delta _{j+2}\delta _{k+1}) \\
& = & (\Phi (\gamma ^{n+1}_{j+1})\delta _{j+2})\delta _{k+1} \\
& = & (\Phi (\gamma ^{n+1}_{j+1}\delta _{j+2}))\delta _{k+1} \\
& = & (\Phi (\gamma ^{n+1}_{j+2}\delta _{j+2}))\delta _{k+1}
\end{array}
\end{displaymath}
and that
\begin{displaymath}
\begin{array}{rcl}
\psi (\gamma ^n_{j+1})\delta _{j+1} & = & (\Phi (\gamma ^{n+1}_{j+2})\delta _{k+1})\delta _{j+1} \\
& = & \Phi (\gamma ^{n+1}_{j+2})(\delta _{k+1} \delta _{j+1}) \\
& = & \Phi (\gamma ^{n+1}_{j+2})(\delta _{j+2}\delta _{k+1}). \\
\end{array}
\end{displaymath}
This confirms that (\ref{equation_compatibility_def_psi}) holds in the case when $j>k$ and concludes our verification of (\ref{equation_compatibility_def_psi}) for any $j$ with $0\leq j<n-1$.

We define $\Psi :\Delta [n-1]\times \Delta [1]\to X$ by letting
\[\Psi (\gamma ^n_j\alpha )=\psi (\gamma ^n_j)\alpha\]
for all $j$ with $0\leq j<n$. The map $\Psi$ is well defined and a simplicial map as $\psi$ satisfies the glueing condition (\ref{equation_compatibility_def_psi}). Thus it remains to argue that
\begin{equation}\label{eq:Phi_degenerate}
\Phi =\Psi \circ (N\sigma _k\times id).
\end{equation}
It suffices to check that the equation holds on the generators $\gamma ^{n+1}_0,\dots ,\gamma ^{n+1}_n$ for the prism $\Delta [n]\times \Delta [1]$.

We use the calculation of $(N\sigma _k\times id)(\gamma ^{n+1}_j)$, $0\leq j\leq n$, above. There are three cases. Either $0\leq j\leq k$, $j=k+1$ or $j>k+1$.

If $0\leq j\leq k$, then
\begin{displaymath}
\begin{array}{rcl}
\Psi \circ (N\sigma _k\times id)(\gamma ^{n+1}_j) & = & \Psi (\gamma ^n_j\sigma _{k+1}) \\
& = & \psi (\gamma ^n_j)\sigma _{k+1} \\
& = & (\Phi (\gamma ^{n+1}_j)\delta _{k+1})\sigma _{k+1} \\
& = & \Phi (\gamma ^{n+1}_j),
\end{array}
\end{displaymath}
which confirms (\ref{eq:Phi_degenerate}) for the generators $\gamma ^{n+1}_0, \dots \gamma ^{n+1}_k$. This is because the vertices of $\Phi (\gamma ^{n+1}_j)$ that are numbered $k+1$ and $k+2$ are equal. Thus the simplex splits off $\sigma _{k+1}$ by \cref{lem:degenerate_part_factorization_through} as $X$ is non-singular. Furthermore, $\delta _{k+1}$ is a section of $\sigma _{k+1}$.

Note that $\Phi (\gamma ^{n+1}_j)$ splits off $\sigma _k$ when $j>k$. This is because the vertices of $\Phi (\gamma ^{n+1}_j)$ that are numbered $k$ and $k+1$ are equal. Thus the simplex splits off $\sigma _k$ by \cref{lem:degenerate_part_factorization_through} as $X$ is non-singular. Furthermore, $\delta _{k+1}$ is a section of $\sigma _k$.

Consider the case when $j=k+1$. We get that
\begin{displaymath}
\begin{array}{rcl}
\Psi \circ (N\sigma _k\times id)(\gamma ^{n+1}_{k+1}) & = & \Psi (\gamma ^n_k\sigma _k) \\
& = & \psi (\gamma ^n_k)\sigma _k \\
& = & (\Phi (\gamma ^{n+1}_k)\delta _{k+1})\sigma _k \\
& = & (\Phi (\gamma ^{n+1}_{k+1})\delta _{k+1})\sigma _k \\
& = & \Phi (\gamma ^{n+1}_{k+1}),
\end{array}
\end{displaymath}
which confirms (\ref{eq:Phi_degenerate}) for the generator $\gamma ^{n+1}_{k+1}$.

Finally, we consider the case when $j>k+1$. Then
\begin{displaymath}
\begin{array}{rcl}
\Psi \circ (N\sigma _k\times id)(\gamma ^{n+1}_j) & = & \Psi (\gamma ^n_{j-1}\sigma _k) \\
& = & \psi (\gamma ^n_{j-1})\sigma _k \\
& = & (\Phi (\gamma ^{n+1}_j)\delta _{k+1})\sigma _k \\
& = & \Phi (\gamma ^{n+1}_j),
\end{array}
\end{displaymath}
which confirms (\ref{eq:Phi_degenerate}) for the generators $\gamma ^{n+1}_{k+2}, \dots ,\gamma ^{n+1}_n$. This concludes our verification of (\ref{eq:Phi_degenerate}). Thus $\Phi$ is a degenerate simplex of $X^{\Delta [1]}$.
\end{proof}

