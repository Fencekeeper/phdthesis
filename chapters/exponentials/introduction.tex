\section{Introduction}
\label{sec:intro_exp}

There are times when one would like to know whether a category behaves similarly, in some sense, to the category of sets and functions. As an example, for homotopy-theoretical purpose the author would like to know whether the endofunctor $-\times \Delta [1]$ of non-singular simplicial sets preserves colimits. Here, $\Delta [1]$ denotes the standard $1$-simplex.

Let $sSet$ denote the category of simplicial sets. The full subcategory $nsSet$ whose objects are the non-singular simplicial sets sits strictly between $sSet$ and the category of ordered simplicial complexes. Despite the fact that non-singular simplicial sets have a natural PL structure \cite[p.~126--127]{WJR13} they almost never appear in the literature, though they do play a role in the book Spaces of PL Manifolds and Categories of Simple Maps by Waldhausen, Jahren and Rognes \cite{WJR13}.

The endofunctor $(-)^K:sSet\to sSet$ is designed so that the Yoneda lemma makes it right adjoint to $-\times K$. Our main result is the following.
\begin{theorem}\label{thm:arbitrary_exponent}
Let $K$ be some simplicial set. Then $X^K$ is non-singular whenever $X$ is.
\end{theorem}
\noindent Part of the author's interest in this result comes from the case when $K$ non-singular. Then the restriction of $(-)^K$ to $nsSet$ corestricts to an endofunctor of non-singular simplicial sets. Moreover, $(-)^K$ viewed as a functor $nsSet\to nsSet$ is right adjoint to the endofunctor $-\times K$ of $nsSet$. This means that we can derive the following consequence of \cref{thm:arbitrary_exponent}.
\begin{corollary}\label{cor:take_product_cocontinous_endofunctor_non-singular}
Taking the product $-\times K:nsSet\to nsSet$ with a non-singular simplicial set $K$ preserves colimits.
\end{corollary}
\noindent In particular, taking the product $-\times \Delta [1]$ with an interval is a cocontinous endofunctor of non-singular simplicial sets.

The case of the interval is not only of practicle concern, but it is also the theoretical focus of this article as it is not hard to argue that \cref{thm:arbitrary_exponent} follows from the following result.
\begin{proposition}\label{prop:standard_1_simplex_as_exponent}
The simplicial set $X^{\Delta [1]}$ is non-singular whenever $X$ is.
\end{proposition}
\noindent The proof of the latter result is the subject of \cref{sec:prism}, whereas \cref{thm:arbitrary_exponent} is derived from \cref{prop:standard_1_simplex_as_exponent} in \cref{sec:arbexp}.

In \cref{sec:app}, we will discuss applications of \cref{thm:arbitrary_exponent} beyond \cref{cor:take_product_cocontinous_endofunctor_non-singular}. We explain how \cref{thm:arbitrary_exponent} follows from \cref{prop:standard_1_simplex_as_exponent} in \cref{sec:arbexp}. Finally, the case of the interval is discussed \cref{sec:prism}.



