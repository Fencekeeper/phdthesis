

\section{Filtered colimits in $nsSet$}
\label{sec:smallness}

Recall from \cref{def:sequence_composition} the notion of sequence in a cocomplete category. A sequence is an example of a functor from a small filtered category. This is because a non-empty ordinal is an example of a (small) filtered category. There is a standard result that makes filtered categories appealing. It says that filtered colimits commute with finite limits. We will use this result in \cref{sec:planes}, which is the most technical part of our procedure to establish $nsSet$ as a model category.
\begin{definition}\label{def:filtered_category}
A category $J$ is \textbf{filtered} if it contains at least one object and satisfies the following two conditions.
\begin{enumerate}
\item{For any two objects $j$ and $j'$ there is a third object $k$ and morphisms $j\to k$ and $j'\to k$.
\begin{displaymath}
\xymatrix@=1em{
j \ar@{-->}[drr] \\
&& k \\
j' \ar@{-->}[urr]
}
\end{displaymath}
}
\item{For any two parallell morphisms $u,v:i\to j$ there is an object $k$ together with a morphism $w:j\to k$ that makes the diamond-shaped diagram
\begin{displaymath}
\xymatrix@=1em{
&& j \ar@{-->}[drr]^w \\
i \ar[drr]_v \ar[urr]^u &&&& k \\
&& j \ar@{-->}[urr]_w
}
\end{displaymath}
commute.
}
\end{enumerate}
\end{definition}
To take advantage of sequences in $sSet$, we will present \cref{lem:filtered_colimits_preservation}, which says that the inclusion $U:nsSet\to sSet$ preserves filtered colimits. In \cref{ch:htythy}, we will use \cref{lem:filtered_colimits_preservation} several times.

In particular, there is a technique by Quillen \cite{Qu67} called the small object argument \cite[Prop.~10.5.16, p.~198]{Hi03}. It enables the construction of functorial factorizations in a cocomplete category $\mathscr{C}$. Namely, the factorizations ought to be as a cofibration followed by a trivial fibration or into a trivial cofibration followed by a fibration in order to confirm the Factorization axiom.

For the factorization technique to work, one lets $A$ be an object in $\mathscr{C}$ that is of technical importance and asks that the covariant hom functor $\mathscr{C} (A,-)$ behaves reasonably with respect to sequences. If it does, then one says, loosely, that $A$ is small. We will state precisely the nature of said behavior in \cref{sec:lifting}. There, we will present a smallness result for non-singular simplicial sets as part of the argument to establish $nsSet$ as a model category. We will use \cref{lem:filtered_colimits_preservation} in this situation as well.

As promised, we present the following result.
\begin{lemma}\label{lem:filtered_colimits_preservation}
The inclusion $U:nsSet\to sSet$ preserves filtered colimits.
\end{lemma}
\begin{proof}
We will prove the claim of \cref{lem:filtered_colimits_preservation} in the following way. Given a functor $F:J\to sSet$ where $J$ is a small filtered category that is such that $F(j)$ is non-singular for each object $j$ in $J$, we will argue that the colimit of $F$ is non-singuar.

Let $Z$ be the colimit of $F$. As colimits in $sSet$ are taken in each degree, we can assume that
\[Z_n=\bigsqcup _{j\in J}F(j)_n/\simeq\]
where $\simeq$ is the equivalence relation generated by a binary relation $\sim$ defined by
\[F(j)\ni x\sim x'\in F(j')\Leftrightarrow \exists u:j\to j':(F(u))(x)=x'.\]
The binary relation $\sim$ is reflexive and transitive, but not necessarily symmetric.

Suppose $z\in Z_n$ not embedded. We will prove that $z$ is degenerate. It will thus follow that $Z$ is non-singular. As $z$ is not embedded there are $k,l\in [n]$ with $k<l$ such that $z\varepsilon _k=z\varepsilon _l$.

Suppose $j_z$ an object of $J$ such that $z$ is in the image of $F(j_z)_n\to Z_n$, meaning that there is a $x\in F(j_z)_n$ such that the map sends $x\mapsto z$. If $x$ is degenerate, then $z$ is. We will consider the case when $x$ is non-degenerate.

Responsible for the assumption that $x\varepsilon _k\simeq x\varepsilon _l$ is a diagram
\begin{displaymath}
\xymatrix@=0.8em{
& j_z \ar@{-}[dr] \\
j_0 \ar@{-}[ur] && j_q \ar@{-}[d] \\
j_1 \ar@{-}[u] \ar@{-}[dr] && j_{q-1} \\
& \dots \ar@{-}[ur]
}
\end{displaymath}
where each of the morphisms can go in either direction and that induces a diagram
\begin{displaymath}
\xymatrix@=0.8em{
& F(j_z)_0 \ar@{-}[dr] \\
F(j_0)_0 \ar@{-}[ur] && F(j_q)_0 \ar@{-}[d] \\
F(j_1)_0 \ar@{-}[u] \ar@{-}[dr] && F(j_{q-1})_0 \\
& \dots \ar@{-}[ur]
}
\end{displaymath}
that connects $x\varepsilon _k$ with $x\varepsilon _l$.

Next, we use that $J$ is a filtered category and a standard argument. By condition $1$ of \cref{def:filtered_category}, there is some object $j_{z0}$ together with morphisms $j_z\to j_{z0}$ and $j_0\to j_{z0}$. In the case when the morphism between $j_z$ and $j_0$ is a morphism $j_z\to j_0$, then by condition $2$ of \cref{def:filtered_category}, we can choose the object $j_{z0}$ and the morphisms above such that the morphism $j_z\to j_{z0}$ is equal to the composite $j_z\to j_0\to j_{z0}$. The case when $j_z$ is instead the target and $j_0$ the source of the morphism between them, is similar.

Similarly to the procedure in the previous paragraph, we can find objects $j_{z1}$, $\dots$, $j_{qz}$ and morphisms as indicated in the diagram
\begin{displaymath}
\xymatrix@=1em{
j_z \ar[dr] \ar@{-}[rr] && j_0 \ar[ld] \ar[dr] \ar@{-}[rr] && \dots \ar@{-}[rr] && j_k \ar[ld] \ar[dr] \ar@{-}[rr] && j_z \ar[ld] \\
& j_{z0} && j_{01} && j_{(k-1)k} && j_{kz}
}
\end{displaymath}
that make the triangles that appear commute. If we continue in this way, namely alternating between invoking the first and the second condition of \cref{def:filtered_category}, then we get a commutative diagram
\begin{displaymath}
\xymatrix@=1em{
j_z \ar[dr] \ar@{-}[rr] && j_0 \ar[ld] \ar[dr] \ar@{-}[rr] && \dots && \dots \ar@{-}[rr] && j_k \ar[ld] \ar[dr] \ar@{-}[rr] && j_z \ar[ld] \\
& j_{z0} \ar[dr] && j_{01} \ar[ld] \ar[dr] &&&& j_{(k-1)k} \ar[ld] \ar[dr] && j_{kz} \ar[ld] \\
&& \dots && \dots && \dots && \dots \\
&&&& \dots \ar[dr] && \dots \ar[ld] \\
&&&&& j
}
\end{displaymath}
in $J$.

Choose an object $j'$ and a morphism $j\to j'$ such that the composites
\[j_z\to j_{z0}\to \cdots \to j\to j'\]
and
\[j_z\to j_{kz}\to \cdots \to j\to j'\]
are equal. Let $y$ be the image of $x$ under the composite
\[F(j_z)\to F(j_{z0})\to \cdots \to F(j)\to F(j').\]
Consequently, we have the equality $y\varepsilon _k=y\varepsilon _l$. This implies that $y$ is degenerate because $F(j')$ is non-singular by assumption. The image of $y$ under $F(j')\to Z$ is $z$, so $z$ is degenerate. This concludes our argument that $Z$ is non-singular.
\end{proof}





