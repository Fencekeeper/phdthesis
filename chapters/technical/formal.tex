

\section{How to build a non-singular simplicial set}
\label{sec:formal}

Any non-singular simplicial set is a colimit over its non-degenerate simplices in the same way that a simplicial set is a colimit over its simplices. This type of viewpoint has proven useful in $sSet$, so we present a proof of a similar statement for $nsSet$ as we are about to establish a model structure on the latter category.

Various simplex categories for a simplicial set $X$ appear in the literature. A common variant is the category $\Delta \downarrow X$ defined thus. Its objects are the representing maps $\bar{x}$ of simplices of $X$. Given simplices $x$ and $y$, say of degree $n$ and $m$, respectively, then the morphisms $\bar{y} \to \bar{x}$ are the commutative triangles
\begin{displaymath}
\xymatrix{
\Delta [m] \ar[dr]_{\bar{y} } \ar[rr]^\alpha && \Delta [n] \ar[ld]^{\bar{x} } \\
& X
}
\end{displaymath}
which is the same as saying that $y=x\alpha$.

By design, the simplicial set $X$ is itself the colimit of the composite
\[\Delta \downarrow X\to \Delta \xrightarrow{\Upsilon } sSet,\]
denoted $\Upsilon _X$. A reference is Lemma 3.1.3. in Hovey's book \cite[p.~75]{Ho99}. Here, the functor $\Upsilon$ is the Yoneda embedding and $\Delta \downarrow X\to \Delta$ is the forgetful functor that sends a representing map $\bar{x} :\Delta [n]\to X$ to the ordinal $[n]$, following Chapter 4 in \cite{FP90}. Viewing $X$ as a colimit over its simplices is a useful technical tool when dealing with simplicial sets.

When $X$ is non-singular, it turns out that $X$ is even a colimit over its non-degenerate simplices. Let $\Upsilon _X'$ be the restriction of $\Upsilon _X$ to the full subcategory $\Delta '\downarrow X$ of $\Delta \downarrow X$ whose objects are the representing maps of the non-degenerate simplices.
\begin{proposition}\label{prop:non-singular_colimit_over_non-degenerate_simplices}
Let $X$ be a simplicial set. If $X$ is non-singular, then it is the colimit (in $sSet$) of $\Upsilon _X'$.
\end{proposition}
\noindent This result is known among users of the category of non-singular simplicial sets. It is presented without proof in \cite[Lemma~3.1.4, p.~76]{Ho99}, but without any assumption on the simplicial set $X$. In that case the statement is wrong, but this is commented on and corrected in the corresponding erratum, which is part of the book's online resources.

Hovey's erratum uses the name \emph{regular simplicial set} for the term non-singular simplicial set. This may be an unfortunate choice as the name \emph{regular simplicial set} seems established as a simplicial set such that each non-degenerate $n$-simplex is attached along its $n$-th face, for each $n\geq 0$. At least, the latter meaning is implied in \cite{FP90}. There, the word regular is seen in connection with regularity of CW-complexes.

Towards proving the proposition, we have the following interesting result.
\begin{lemma}
\label{lem:non-degenerate_simplices_reflective_subcategory_non-singular}
Let $X$ be a non-singular simplicial set. The inclusion
\[i:\Delta '\downarrow X\to \Delta \downarrow X\]
has a retraction.
\end{lemma}
\begin{proof}
We explain that the rule $\bar{x} \mapsto \overline{x^\sharp }$ defines a retraction $r$ of the inclusion $i$.

On morphisms $\bar{y} \xrightarrow{(x,\alpha )} \bar{x}$, where $y$ and $x$ are of degree $m$ and $n$, respectively, we define $r$ thus. Suppose $y^\sharp$ and $x^\sharp$ of degree $k$ and $l$, respectively. Now we need a choice of a section of the degenerate part $x^\flat$ of each object $\bar{x}$ of $\Delta \downarrow X$. The choice does not matter for our purposes, although there are systematic choices of sections of degeneracy operators, for example the maximal section $(-)^\perp$ \cite[p.~136]{FP90}.

Next, we expand the diagram above to
\begin{displaymath}
\xymatrix{
\Delta [k] \ar@/_2pc/[dd]_1 \ar[d]^{(y^\flat )^\perp } && \Delta [l] \ar[d]_{(x^\flat )^\perp } \ar@/^2pc/[dd]^1 \\
\Delta [m] \ar[d]^{y^\flat } \ar[rr]^\alpha && \Delta [n] \ar[d]_{x^\flat } \\
\Delta [k] \ar[dr]_{\overline{y^\sharp } } && \Delta [l] \ar[ld]^{\overline{x^\sharp } } \\
& X
}
\end{displaymath}
where we have displayed our choices of sections to the degenerate parts. The diagram gives rise to the morphism
\begin{displaymath}
\xymatrix{
\Delta [k] \ar[dr]_{\overline{y^\sharp } } \ar[rr]^{x^\flat \circ \alpha \circ (y^\flat )^\perp } && \Delta [l] \ar[ld]^{\overline{x^\sharp } } \\
& X
}
\end{displaymath}
from $\overline{y^\sharp }$ to $\overline{x^\sharp }$, which can be denoted $r(x,\alpha )$. The triangle commutes, so
\[\mu =x^\flat \circ \alpha \circ (y^\flat )^\perp\]
must be a face operator as $y^\sharp$ is non-degenerate.

Different choices of sections of the degenerate parts could perhaps lead to different morphisms $\overline{y^\sharp } \to \overline{x^\sharp }$, but not if $X$ is non-singular. In that case, the simplex $y^\sharp$ is a face of $x^\sharp$ in a unique way. Moreover, the rule of sending the given morphism $\bar{y} \xrightarrow{(x,\alpha )} \bar{x} $ to $\overline{y^\sharp } \xrightarrow{(x^\sharp ,\mu )} \overline{x^\sharp }$ respects composition when $X$ is non-singular for the same reason. In other words, we get a retraction
\[r:\Delta \downarrow X\to \Delta '\downarrow X\]
of the inclusion $i$.
\end{proof}
\noindent In \cref{sec:simplexcat}, we discuss various simplex categories and their relations.\\ \cref{lem:non-degenerate_simplices_reflective_subcategory_non-singular} provides an example of such a relation.

We are ready to prove the proposition. The proof consists of a recognition that the relationship between $\Delta \downarrow X$ and $\Delta '\downarrow X$ is improved over the general case when $X$ is non-singular.
\begin{proof}[Proof of \cref{prop:non-singular_colimit_over_non-degenerate_simplices}.]
Let $r$ be the retraction of
\[i:\Delta '\downarrow X\to \Delta \downarrow X\]
from \cref{lem:non-degenerate_simplices_reflective_subcategory_non-singular}.

Notice that there is a map $\bar{x} \to ir(\bar{x} )$, defined as the commutative triangle
\begin{displaymath}
\xymatrix{
\Delta [n] \ar[dr]_{\bar{x} } \ar[rr]^{x^\flat } && \Delta [l] \ar[ld]^{\overline{x^\sharp } } \\
& X
}
\end{displaymath}
in the case when $x$ is of degree $n$ and when $x^\sharp$ is of degree $l$. The diagram
\begin{equation}
\label{eq:diagram_proof_lem_non-degenerate_simplices_reflective_subcategory}
\begin{gathered}
\xymatrix{
\Delta [m] \ar@/_1pc/[dddr]_{\bar{y} } \ar[dr]_{y^\flat } \ar[rr]^{\alpha } && \Delta [n] \ar@{-}@/^/[d] \ar[dr]^{x^\flat } \\
& \Delta [k] \ar@/^/[dd]_{\overline{y^\sharp } } \ar[rr]^(.3)\mu & \ar@/^/[ldd]^(.35){\bar{x} } & \Delta [l] \ar@/^1pc/[lldd]^{\overline{x^\sharp } } \\
\\
& X
}
\end{gathered}
\end{equation}
commutes if the top square commutes. Furthermore, the top square of (\ref{eq:diagram_proof_lem_non-degenerate_simplices_reflective_subcategory}) commutes as $\overline{x^\sharp }$ is a monomorphism. Thus the map $\bar{x} \to ir(\bar{x} )$ is in fact natural.

That $X$ is the colimit of $\Upsilon _X$ is the same as saying that the cocone $\Upsilon _X\Rightarrow \underline{X}$ that arises from the definition of $\Delta \downarrow X$ is universal. The symbol $\underline{X}$ denotes the the functor $\Delta \downarrow X\to sSet$ that sends each object to $X$ and each morphism to the identity $1_X$. We refer to $\underline{X}$ as the \textbf{constant diagram} at $X$. Sometimes this language and notation is convenient.

The notation and terminology of the previous paragraph is more or less taken from Section 2.6 in May's book on algebraic topology \cite{Ma99}. There, the notion of (co)cone is of course present to describe (co)limits, although the term (co)cone is not used. (Co)limits are, however, referred to as universal (co)cones in \cite{Bo94}.

Note that the unit of the adjunction
\[r:\Delta \downarrow X\rightleftarrows \Delta '\downarrow X:i\]
yields a natural transformation $\Upsilon _X\Rightarrow \Upsilon _X\circ ir$. Recall that $r$ is a retraction of $i$. Let $X'$ denote the colimit of $\Upsilon _X'$. To prove \cref{prop:non-singular_colimit_over_non-degenerate_simplices} is to prove universality of the cocone $\Upsilon _X'\Rightarrow \underline{X}$ that arises from the universal cocone $\Upsilon _X\Rightarrow \underline{X}$.

Combine the universal cocone $\Upsilon _X\Rightarrow \underline{X}$ with $i$ to obtain a triangle
\begin{displaymath}
\xymatrix{
\Upsilon _X' \ar@{=>}[dr] \ar@{=>}[rr] && \underline{X} \\
& \underline{X'} \ar@{==>}[ur]
}
\end{displaymath}
where the dashed natural transformation appears because $X'$ is the colimit of $\Upsilon _X'=\Upsilon _X\circ i$. We will prove that the canonical map $X'\to X$ is an isomorphism.

In turn, we get the diagram
\begin{displaymath}
\xymatrix{
\Upsilon _X \ar@{=>}[d] \ar@{=>}[r] & \Upsilon _X'\circ r \ar@{=>}[dr] \ar@{=>}[rr] && \underline{X} \\
\underline{X} \ar@{==>}[rr] && \underline{X'} \ar@{=>}[ur]
}
\end{displaymath}
which means that we now have a composite $X\to X'\to X$. Here, we have used the natural transformation $\Upsilon _X\Rightarrow \Upsilon _X\circ ir=\Upsilon _X'\circ r$ that arises from the unit of $(r,i)$.

The composite
\[\Upsilon _X(\bar{x} )\to \Upsilon _X'\circ r(\bar{x} )\to \underline{X} (\bar{x} )\]
is precisely the composite
\[\Delta [n]\xrightarrow{x^\flat } \Delta [l]\xrightarrow{\overline{x^\sharp } } X\]
if $x$ is of degree $n$ and $x^\sharp$ is of degree $l$. In other words, the cocone
\[\Upsilon _X\Rightarrow \Upsilon _X'\circ r\Rightarrow \underline{X}\]
is actually the universal one. Finally, by applying $i$ once more, we obtain
\begin{displaymath}
\xymatrix{
&& \Upsilon _X' \ar@{=>}[lld] \ar@{=>}[ldd] \ar@{=>}[ddr] \ar@{=>}[drr] \\
\underline{X'} \ar@{==>}[dr] &&&& \underline{X} \\
& \underline{X} \ar@{=>}[rr] && \underline{X'} \ar@{=>}[ur]
}
\end{displaymath}
where the two cocones with apex $X'$ are universal. Thus arises a commutative diagram
\begin{displaymath}
\xymatrix{
X' \ar[r] \ar@/^1pc/[rr]^1 & X \ar[r] \ar@/_1pc/[rr]_1 & X' \ar[r] & X
}
\end{displaymath}
showing that $X'\to X$ is an isomorphism as announced.
\end{proof}
